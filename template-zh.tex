%% start of file `template-zh.tex'.
%% Copyright 2006-2015 Xavier Danaux (xdanaux@gmail.com), 2020-2022 moderncv maintainers (github.com/moderncv).
%
% This work may be distributed and/or modified under the
% conditions of the LaTeX Project Public License version 1.3c,
% available at http://www.latex-project.org/lppl/.


\documentclass[11pt,a4paper,sans]{moderncv}   % possible options include font size ('10pt', '11pt' and '12pt'), paper size ('a4paper', 'letterpaper', 'a5paper', 'legalpaper', 'executivepaper' and 'landscape') and font family ('sans' and 'roman')


% moderncv 主题
\moderncvstyle{casual}                        % 选项参数是 ‘casual’, ‘classic’, ‘oldstyle’ 和 ’banking’
\moderncvcolor{blue}                          % 选项参数是 ‘blue’ (默认)、‘orange’、‘green’、‘red’、‘purple’ 和 ‘grey’
%\renewcommand{\familydefault}{\sfdefault}         % to set the default font; use '\sfdefault' for the default sans serif font, '\rmdefault' for the default roman one, or any tex font name
%\nopagenumbers{}                                  % uncomment to suppress automatic page numbering for CVs longer than one page



% 调整页面出血
\usepackage[scale=0.75]{geometry}
%\setlength{\hintscolumnwidth}{3cm}           % 如果你希望改变日期栏的宽度
\setlength{\footskip}{136.00005pt}                 % depending on the amount of information in the footer, you need to change this value. comment this line out and set it to the size given in the warning
%\setlength{\hintscolumnwidth}{3cm}                % if you want to change the width of the column with the dates
%\setlength{\makecvheadnamewidth}{10cm}            % for the 'classic' style, if you want to force the width allocated to your name and avoid line breaks. be careful though, the length is normally calculated to avoid any overlap with your personal info; use this at your own typographical risks...
% font loading
% for luatex and xetex, do not use inputenc and fontenc
% see https://tex.stackexchange.com/a/496643
\ifxetexorluatex
  \usepackage{fontspec}
  \usepackage{unicode-math}
  \defaultfontfeatures{Ligatures=TeX}
  \setmainfont{Sarasa Mono SC}
  \setsansfont{Noto Sans CJK SC}
  \setmonofont{Sarasa Mono SC}
  \setmathfont{Latin Modern Math} 
  %\usepackage[slantfont,boldfont]{xeCJK}
  %\defaultfontfeatures{Mapping=tex-text}
  \usepackage{xcolor}
  \XeTeXlinebreaklocale "zh"
  \XeTeXlinebreakskip = 0pt plus 1pt minus 0.1pt
\else
  \usepackage[utf8]{inputenc}
  \usepackage{CJKutf8}
  \usepackage[T1]{fontenc}
  \usepackage{lmodern}
\fi

% document language
\usepackage[english]{babel}  % FIXME: using spanish breaks moderncv

% 个人信息
\firstname{小龙}
\familyname{李}
\born{4 July 1776}                                 % optional, remove / comment the line if not wanted
\title{简历题目 (可选项)}                      % 可选项、如不需要可删除本行
\address{街道及门牌号}{邮编及城市}             % 可选项、如不需要可删除本行
\phone[mobile]{+1~(234)~567~890}                         % 可选项、如不需要可删除本行
\phone[fixed]{+2~(345)~678~901}                          % 可选项、如不需要可删除本行
\fax{+3~(456)~789~012}                            % 可选项、如不需要可删除本行
\email{xiaolong@li.com.cn}                    % 可选项、如不需要可删除本行
\homepage{www.xialongli.com}                  % 可选项、如不需要可删除本行
\extrainfo{附加信息 (可选项)}                  % 可选项、如不需要可删除本行
\photo[64pt][0.4pt]{picture}                  % ‘64pt’是图片必须压缩至的高度、‘0.4pt‘是图片边框的宽度 (如不需要可调节至0pt)、’picture‘ 是图片文件的名字;可选项、如不需要可删除本行
\quote{引言(可选项)}                           % 可选项、如不需要可删除本行

% 显示索引号;仅用于在简历中使用了引言
%\makeatletter
%\renewcommand*{\bibliographyitemlabel}{\@biblabel{\arabic{enumiv}}}
%\makeatother

% 分类索引
%\usepackage{multibib}
%\newcites{book,misc}{{Books},{Others}}
%----------------------------------------------------------------------------------
%            内容
%----------------------------------------------------------------------------------
\begin{document}
\ifxetexorluatex
\else
\begin{CJK*}{UTF8}{gbsn}                          % to typeset your resume in Chinese using CJK
\fi
%-----       resume       ---------------------------------------------------------
\makecvtitle

\section{教育背景}
\cventry{年 -- 年}{学位}{院校}{城市}{\textit{成绩}}{说明}  % 第3到第6编码可留白
\cventry{年 -- 年}{学位}{院校}{城市}{\textit{成绩}}{说明}

\section{毕业论文}
\cvitem{题目}{\emph{题目}}
\cvitem{导师}{导师}
\cvitem{说明}{\small 论文简介}

\section{工作背景}
\subsection{专业}
\cventry{年 -- 年}{职位}{公司}{城市}{}{不超过1--2行的概况说明\newline{}%
工作内容:%
\begin{itemize}%
\item 工作内容 1;
\item 工作内容 2、 含二级内容:
  \begin{itemize}%
  \item 二级内容 (a);
  \item 二级内容 (b)、含三级内容 (不建议使用);
    \begin{itemize}
    \item 三级内容 i;
    \item 三级内容 ii;
    \item 三级内容 iii;
    \end{itemize}
  \item 二级内容 (c);
  \end{itemize}
\item 工作内容 3。
\end{itemize}}
\cventry{年 -- 年}{职位}{公司}{城市}{}{说明行1\newline{}说明行2}
\subsection{其他}
\cventry{年 -- 年}{职位}{公司}{城市}{}{说明}

\section{语言技能}
\cvitemwithcomment{语言 1}{水平}{评价}
\cvitemwithcomment{语言 2}{水平}{评价}
\cvitemwithcomment{语言 3}{水平}{评价}

\section{计算机技能}
\cvdoubleitem{类别 1}{XXX, YYY, ZZZ}{类别 4}{XXX, YYY, ZZZ}
\cvdoubleitem{类别 2}{XXX, YYY, ZZZ}{类别 5}{XXX, YYY, ZZZ}
\cvdoubleitem{类别 3}{XXX, YYY, ZZZ}{类别 6}{XXX, YYY, ZZZ}

\section{个人兴趣}
\cvitem{爱好 1}{\small 说明}
\cvitem{爱好 2}{\small 说明}
\cvitem{爱好 3}{\small 说明}

\section{其他 1}
\cvlistitem{项目 1}
\cvlistitem{项目 2}
\cvlistitem{项目 3}

\renewcommand{\listitemsymbol}{-}             % 改变列表符号

\section{其他 2}
\cvlistdoubleitem{项目 1}{项目 4}
\cvlistdoubleitem{项目 2}{项目 5\cite{book1}}
\cvlistdoubleitem{项目 3}{}

% 来自BibTeX文件但不使用multibib包的出版物
%\renewcommand*{\bibliographyitemlabel}{\@biblabel{\arabic{enumiv}}}% BibTeX的数字标签
\nocite{*}
\bibliographystyle{plain}
\bibliography{publications}                    % 'publications' 是BibTeX文件的文件名

% 来自BibTeX文件并使用multibib包的出版物
%\section{出版物}
%\nocitebook{book1,book2}
%\bibliographystylebook{plain}
%\bibliographybook{publications}               % 'publications' 是BibTeX文件的文件名
%\nocitemisc{misc1,misc2,misc3}
%\bibliographystylemisc{plain}
%\bibliographymisc{publications}               % 'publications' 是BibTeX文件的文件名

\clearpage
\ifxetexorluatex
\else
\clearpage\end{CJK*}                              % if you are typesetting your resume in Chinese using CJK; the \clearpage is required for fancyhdr to work correctly with CJK, though it kills the page numbering by making \lastpage undefined
\fi
\end{document}


%% 文件结尾 `template-zh.tex'.