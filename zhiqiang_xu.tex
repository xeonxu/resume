%% start of file `template-zh.tex'.
%% Copyright 2006-2012 Xavier Danaux (xdanaux@gmail.com).
%
% This work may be distributed and/or modified under the
% conditions of the LaTeX Project Public License version 1.3c,
% available at http://www.latex-project.org/lppl/.


\documentclass[12pt,a4paper,sans]{moderncv}   % possible options include font size ('10pt', '11pt' and '12pt'), paper size ('a4paper', 'letterpaper', 'a5paper', 'legalpaper', 'executivepaper' and 'landscape') and font family ('sans' and 'roman')
\usepackage{etoolbox}
\makeatletter
% 字符编码
\usepackage[utf8]{inputenc}                   % 替换你正在使用的编码
% 使用xelatex,无需以下设置
% \usepackage{CJKutf8}

\@ifpackageloaded{tex4ht}{
\usepackage{devng4ht}
}{%
\usepackage{fontspec,xunicode}
\setmainfont{Tahoma}
\usepackage[slantfont,boldfont]{xeCJK}
\setCJKmainfont{Adobe Song Std L}
\defaultfontfeatures{Mapping=tex-text}
\XeTeXlinebreaklocale "zh"
\XeTeXlinebreakskip = 0pt plus 1pt minus 0.1pt
}
\usepackage{xcolor}                 % replace by the encoding you are using

% moderncv 主题
\moderncvstyle{classic}                        % 选项参数是 ‘casual’, ‘classic’, ‘oldstyle’ 和 ’banking’
\moderncvcolor{orange}                          % 选项参数是 ‘blue’ (默认)、‘orange’、‘green’、‘red’、‘purple’ 和 ‘grey’
%\nopagenumbers{}                             % 消除注释以取消自动页码生成功能

% 调整页面
\usepackage[scale=0.75]{geometry}
%\setlength{\hintscolumnwidth}{3cm}           % 如果你希望改变日期栏的宽度
% 解决XeLaTex编译时的编码警告
\usepackage[unicode,pdfencoding=auto]{hyperref}

% 个人信息
\firstname{徐}
\familyname{至强}
\title{个人简历}                      % 可选项、如不需要可删除本行
\address{松江区文翔路3088弄463号402室}{201616,上海}             % 可选项、如不需要可删除本行
\mobile{+86 13585618661}                         % 可选项、如不需要可删除本行
\phone{+86 21 67663700}                          % 可选项、如不需要可删除本行
%% \fax{+3~(456)~789~012}                            % 可选项、如不需要可删除本行
\email{xeonxu@gmail.com}                    % 可选项、如不需要可删除本行
\homepage{blog.xeonxu.info}                  % 可选项、如不需要可删除本行
\extrainfo{www.github.com/xeonxu}                  % 可选项、如不需要可删除本行
\photo[80pt][0.4pt]{picture.jpg}                  % ‘64pt’是图片必须压缩至的高度、‘0.4pt‘是图片边框的宽度 (如不需要可调节至0pt)、’picture‘ 是图片文件的名字;可选项、如不需要可删除本行
\quote{Work as a hacker. Hack as an artist.}                 % optional, remove the line if not wante

% 显示索引号;仅用于在简历中使用了引言
%\makeatletter
%\renewcommand*{\bibliographyitemlabel}{\@biblabel{\arabic{enumiv}}}
%\makeatother

% 分类索引
%\usepackage{multibib}
%\newcites{book,misc}{{Books},{Others}}
%----------------------------------------------------------------------------------
%            内容
%----------------------------------------------------------------------------------
\begin{document}
% \begin{CJK}{UTF8}{gbsn}                       % 详情参阅CJK文件包
\maketitle

\section{教育背景}
\cventry{2002 -- 2006}{交通工程学士}{河海大学}{南京}{}{}  % 第3到第6编码可留白

% \section{毕业论文}
% \cvitem{题目}{\emph{地铁机车运行信息收发报模拟器的实现}}
% \cvitem{导师}{徐鹏}

\section{工作背景}
\cventry{2010 -- 至今}{嵌入式软件工程师}{上海斐讯数据通信技术有限公司}{上
  海}{}{手机研发\newline{}
主要工作:
\begin{itemize}
\item MStar功能机项目。FSB01,FSE71,FSB107,FSB109,FSB304。
  \begin{itemize}
  \item 界面修改,三方软件移植以及解决Bug
  \item 编写自动发布脚本
  \end{itemize}
\item MTK功能机项目。FMB304。
  \begin{itemize}
  \item 界面修改,功能定制,解决Bug
  \end{itemize}
\item Android智能手机项目。在智能手机项目中,本人转为驱动工程师。目前
  为止,在项目中主职电源管理模块的开发,同时也负责部分外围器件的驱动开
  发。
  \begin{itemize}
  \item FWS920,FWS710+,基于Marvell PXA968平台。
  \item FWS710EU,基于Qualcomm MSM7x27平台。
  \item FPAD307,i813w,k390w/v,i508d,基于Qualcomm MSM8x25平台。
  \item i813wa,基于Qualcomm MSM8x25Q平台。
  \item 制作基于windows和linux系统的高通MP编译环境一键安装配置工具
  \item 制作各类项目工具
  \end{itemize}
\end{itemize}
}
\cventry{2009 -- 2010}{嵌入式软件工程师}{上海品酷网络科技有限公司}{上海}{}{开发动态手机主题及手机特效(iShow主题系统)\newline{}
  主要工作:
  \begin{itemize}
  \item 开发基于mtk平台手机的动态主题和菜单特效。
    \begin{itemize}
    \item 宫格类主题,圆环类主题
    \item 飘带效果
    \item 蒲公英效果
    \item 生产自动化脚本工具
    \end{itemize}
  \item 改进及完善手机端的下载程序,支持断点续传。使用户能够通过下载器
    方便的从网络上下载相应的主题包文件,而扩展手机界面的表现形式。
  \end{itemize}
}
\cventry{2008 -- 2009}{嵌入式软件工程师}{上海腾龙软件公司}{上海市}{}{参与几个对日项
  目,主要是嵌入式平台的开发。
  主要工作:
  \begin{itemize}
  \item 日立公司的数字电视项目
    \begin{itemize}
    \item 界面的开发及上部逻辑的工作
    \item 编写文档
    \item 编码测试
    \end{itemize}
  \item vxworks到Linux的移植项目。改进开发环境,快速解决问题。
  \item openvg移植项目接口
    \begin{itemize}
    \item 主体程序移植
    \item 单元测试程序编写及测试
    \end{itemize}
  \end{itemize}
}
\cventry{2008 -- 2008}{Java工程师}{京瓷公司Mita分公司}{大阪市}{}{作为协力会员在京瓷公司,使用Java语言,协助开发用于医院的纸质文档的电子化解决方案KMCapture Solution。\newline{}
  主要工作:
  \begin{itemize}
  \item 担当模块为Controller,Facade以及部分的Storage。
  \item 编写软件设计书
  \item 编码及测试
  \end{itemize}
}
\cventry{2007 -- 2008}{嵌入式软件工程师}{日立制造所}{日立市}{}{作为协力会员为日立公司协助开发基于ARM处理器的车载项目AVCCore。该项目后来为凯迪拉克CTS08,09车型的车载音响。\newline{}
  主要工作:
  \begin{itemize}
  \item iPod集成应用。负责车载电脑中iPod Controller和iPod CoreApp开发和维护
  \item 负责车载电脑的音频文件头解析处理的开发及维护
  \item 文档的编写及维护
  \end{itemize}
}
\cventry{2006 -- 2007}{Java工程师}{南大腾龙}{南京}{}{从事Java相关的测试和开发工作\newline{}
  主要工作:
  \begin{itemize}
  \item 为JUSTSYSTEM开发的xfyuass项目
    \begin{itemize}
    \item 编写单元测试代码
    \item 集成化测试整合
    \item 编写文档
    \end{itemize}
  \item 为三得利公司开发自动售卖机在线维护管理系统的vvv项目
    \begin{itemize}
    \item 编写数据库的处理代码
    \item 编写单元测试代码
    \item 编写自动化部署脚本
    \item 编写文档
    \end{itemize}
  \end{itemize}
}%

\section{语言技能}
\cvitemwithcomment{英语}{CET4}{一般}
\cvitemwithcomment{日语}{三级}{一般}

\section{计算机技能}
\cvdoubleitem{C语言}{熟练}{Java}{一般}
\cvdoubleitem{Linux/Unix}{熟练}{硬件方面}{基本概念}
\cvdoubleitem{脚本}{Bash}{编辑器}{Emacs}

\section{个人兴趣}
\cvitem{电影}{\small 偏爱纪录片以及科幻片}
\cvitem{看书}{\small 80\%为计算机相关图书,偶尔看些小说和项目管理类}
\cvitem{体育}{\small F1,羽毛球}
\cvitem{开源项目}{\small KOReader,twip}

% \section{开源项目}
% \cvlistitem{twip:添加proxy支持}
% \cvlistitem{images2bin}
% \cvlistitem{rasproxy}

\renewcommand{\listitemsymbol}{-}             % 改变列表符号

% \section{其他 2}
% \cvlistdoubleitem{项目 1}{项目 4}
% \cvlistdoubleitem{项目 2}{项目 5\cite{book1}}
% \cvlistdoubleitem{项目 3}{}

% 来自BibTeX文件但不使用multibib包的出版物
%\renewcommand*{\bibliographyitemlabel}{\@biblabel{\arabic{enumiv}}}% BibTeX的数字标签
\nocite{*}
\bibliographystyle{plain}
\bibliography{publications}                    % 'publications' 是BibTeX文件的文件名

% 来自BibTeX文件并使用multibib包的出版物
% \section{出版物}
% \nocitebook{book1,book2}
% \bibliographystylebook{plain}
%\bibliographybook{publications}               % 'publications' 是BibTeX文件的文件名
%\nocitemisc{misc1,misc2,misc3}
%\bibliographystylemisc{plain}
%\bibliographymisc{publications}               % 'publications' 是BibTeX文件的文件名

% \clearpage\end{CJK}
\end{document}


%% 文件结尾 `template-zh.tex'.