%% start of file `template-zh.tex'.
%% Copyright 2006-2012 Xavier Danaux (xdanaux@gmail.com).
%
% This work may be distributed and/or modified under the
% conditions of the LaTeX Project Public License version 1.3c,
% available at http://www.latex-project.org/lppl/.


\documentclass[12pt,a4paper,sans]{moderncv}   % possible options include font size ('10pt', '11pt' and '12pt'), paper size ('a4paper', 'letterpaper', 'a5paper', 'legalpaper', 'executivepaper' and 'landscape') and font family ('sans' and 'roman')
\usepackage{etoolbox}
\makeatletter
% 字符编码
\usepackage[utf8]{inputenc}                   % 替换你正在使用的编码
% 使用xelatex,无需以下设置
% \usepackage{CJKutf8}

\@ifpackageloaded{tex4ht}{
\usepackage{devng4ht}
}{%
\usepackage{fontspec,xunicode}
\setmainfont{Tahoma}
\usepackage[slantfont,boldfont]{xeCJK}
\setCJKmainfont{STXihei}
\defaultfontfeatures{Mapping=tex-text}
\XeTeXlinebreaklocale "zh"
\XeTeXlinebreakskip = 0pt plus 1pt minus 0.1pt
}
\usepackage{xcolor}                 % replace by the encoding you are using

% moderncv 主题
\moderncvstyle{classic}                        % 选项参数是 ‘casual’, ‘classic’, ‘oldstyle’ 和 ’banking’
\moderncvcolor{green}                          % 选项参数是 ‘blue’ (默认)、‘orange’、‘green’、‘red’、‘purple’ 和 ‘grey’
%\nopagenumbers{}                             % 消除注释以取消自动页码生成功能

% 调整页面
\usepackage[scale=0.75]{geometry}
\setlength{\hintscolumnwidth}{4cm}           % 如果你希望改变日期栏的宽度
% 解决XeLaTex编译时的编码警告
\usepackage[unicode,pdfencoding=auto]{hyperref}

% 个人信息
\firstname{徐}
\familyname{至强}
\title{软件工程师}                      % 可选项、如不需要可删除本行
%\address{松江区文翔路3088弄463号402室}{201616,上海}             % 可选项、如不需要可删除本行
\extrainfo{出生:1983.11 年龄:31 籍贯:甘肃  定西}
\mobile{+86 13585618661}                         % 可选项、如不需要可删除本行
\phone{+86 21 67663700}                          % 可选项、如不需要可删除本行
%% \fax{+3~(456)~789~012}                            % 可选项、如不需要可删除本行
\email{xeonxu@gmail.com}                    % 可选项、如不需要可删除本行
\homepage{blog.xeonxu.info}                  % 可选项、如不需要可删除本行
\social[github]{xeonxu}                  % 可选项、如不需要可删除本行
\social[twitter]{xeonxu}                  % 可选项、如不需要可删除本行
\photo[80pt][0.8pt]{picture.jpg}                  % ‘64pt’是图片必须压缩至的高度、‘0.4pt‘是图片边框的宽度 (如不需要可调节至0pt)、’picture‘ 是图片文件的名字;可选项、如不需要可删除本行
\quote{Work as a hacker. Hack as an artist.}                 % optional, remove the line if not want

% 显示索引号;仅用于在简历中使用了引言
%\makeatletter
%\renewcommand*{\bibliographyitemlabel}{\@biblabel{\arabic{enumiv}}}
%\makeatother

% 分类索引
%\usepackage{multibib}
%\newcites{book,misc}{{Books},{Others}}
%----------------------------------------------------------------------------------
%            内容
%----------------------------------------------------------------------------------
\begin{document}
% \begin{CJK}{UTF8}{gbsn}                       % 详情参阅CJK文件包
\maketitle

\section{教育背景}
\cventry{2002.09 -- 2006.06}{交通工程学士学位}{河海大学}{南京}{}{}  % 第3到第6编码可留白

% \section{毕业论文}
% \cvitem{题目}{\emph{地铁机车运行信息收发报模拟器的实现}}
% \cvitem{导师}{徐鹏}

\section{工作背景}
\cventry{2010.06 -- 至今}{嵌入式软件工程师}{上海斐讯数据通信技术有限公司}{上
  海市}{}{工作内容:手机研发\newline{}
主要工作:
\begin{itemize}
 \item Android智能手机项目
  \begin{itemize}
  \item Marvell PXA968平台,传感器,TP以及电源管理芯片驱动。开机logo编译脚本及效果优化。
  \item Qualcomm MSM7x27/MSM8x25/MSM8x25Q平台。主要负责电源管理模块,编译环境优化,地磁传感器调试方法优化,工厂用TP固件升级方案,基于SMB358双路充电的研发,内置U盘镜像自动生成脚本,制定自适应EMMC容量调整内置U盘容量方案。
  \item Qualcomm MSM8x10/8x12/8x26/8x16平台。电源管理模块,电池建模等。
  \end{itemize}
\item MStar及MTK功能机项目
   \begin{itemize}
  \item 界面修改,功能添加
  \item 自动发布脚本
  \end{itemize}
\end{itemize}
}
\cventry{2009.10 -- 2010.06}{嵌入式软件工程师}{上海品酷网络科技有限公司}{上海市}{}{工作内容:开发动态手机主题及手机特效(iShow主题系统)\newline{}
  主要工作:
  \begin{itemize}
  \item 开发基于mtk平台手机的动态主题和菜单特效
    \begin{itemize}
    \item 宫格类主题,圆环类主题
    \item 可无极变色到飘带效果动态屏保
    \item 可区分昼夜的蒲公英效果动态屏保
    \item 编写研发自动化脚本工具
    \end{itemize}
  \item 改进完善手机端的下载程序,支持断点续传。
  \end{itemize}
}
\cventry{2008.11 -- 2009.10}{嵌入式软件工程师}{上海腾龙软件公司}{上海市}{}{工作内容:对日项
  目,主要是嵌入式平台的开发。\newline{}
  主要工作:
  \begin{itemize}
  \item 日立公司的数字电视项目
    \begin{itemize}
    \item 界面开发及Bug修正
    \item 文档维护
    \item 编码测试
    \end{itemize}
  \item vxworks到Linux的移植项目。改进开发环境,快速解决问题。
  \item openvg移植项目接口
    \begin{itemize}
    \item 主体程序移植
    \item 单元测试程序编写及测试
    \end{itemize}
  \end{itemize}
}
\cventry{2008.06 -- 2008.11}{Java工程师}{京瓷公司Mita分公司}{大阪市}{}{工作内容:使用Java语言,在京瓷公司协助其开发用于医院的纸质文档的电子化解决方案KMCapture Solution。\newline{}
  主要工作:
  \begin{itemize}
  \item 主要负责Controller,Facade以及部分的Storage模块。
  \item 编写软件设计书
  \item 编码及测试
  \end{itemize}
}
\cventry{2007.05 -- 2008.06}{嵌入式软件工程师}{日立制造所}{日立市}{}{工作内容:在日立公司协助其开发基于ARM处理器的车载项目AVCCore。该项目后来为凯迪拉克CTS08,09车型的车载音响。\newline{}
  主要工作:
  \begin{itemize}
  \item iPod集成应用。负责车载电脑中iPod Controller和iPod CoreApp开发和维护
  \item 负责车载电脑的音频文件头解析处理的开发及维护
  \item 文档编写及维护
  \end{itemize}
}
\cventry{2006.07 -- 2007.05}{Java工程师}{南大腾龙}{南京市}{}{工作内容:为JUSTSYSTEM开发的xfyuass以及三得利公司开发自动售卖机在线维护管理系统。\newline{}
  主要工作:
  \begin{itemize}
  \item 编写数据库的处理代码
  \item 自动化部署脚本编写
  \item 测试代码编写
  \item 文档维护
  \end{itemize}
}%
\section{语言技能}
\cvitemwithcomment{英语}{CET4}{一般}
\cvitemwithcomment{日语}{三级}{一般}

\section{技能概述}
\cvdoubleitem{C语言}{熟练}{Java}{略懂}
\cvdoubleitem{Objective-C}{学习中}{Swift}{学习中}
\cvdoubleitem{Linux/Unix}{熟练}{硬件}{简单焊接}
\cvdoubleitem{脚本}{Bash}{编辑器}{Emacs/Vim}

\section{个人兴趣}
\cvitem{电影}{\small 纪录片,科幻片}
\cvitem{看书}{\small 技术书籍,小说}
\cvitem{体育}{\small 滑板运动,F1}

\section{开源项目}
\cvlistitem{twip:添加proxy支持}
\cvlistitem{ChinaDNS-C: Bug修复及tomato编译支持}
\cvlistitem{Koreader:Bug修复及编译速度优化}
\cvlistitem{battery\_analyzer:电池电量表自动测算}
\cvlistitem{vim\_configs:维护的用于公司工作的vim配置}
% \cvlistitem{images2bin}
% \cvlistitem{rasproxy}

\renewcommand{\listitemsymbol}{-}             % 改变列表符号

% \section{其他 2}
% \cvlistdoubleitem{项目 1}{项目 4}
% \cvlistdoubleitem{项目 2}{项目 5\cite{book1}}
% \cvlistdoubleitem{项目 3}{}

% 来自BibTeX文件但不使用multibib包的出版物
%\renewcommand*{\bibliographyitemlabel}{\@biblabel{\arabic{enumiv}}}% BibTeX的数字标签
\nocite{*}
\bibliographystyle{plain}
\bibliography{publications}                    % 'publications' 是BibTeX文件的文件名

% 来自BibTeX文件并使用multibib包的出版物
% \section{出版物}
% \nocitebook{book1,book2}
% \bibliographystylebook{plain}
%\bibliographybook{publications}               % 'publications' 是BibTeX文件的文件名
%\nocitemisc{misc1,misc2,misc3}
%\bibliographystylemisc{plain}
%\bibliographymisc{publications}               % 'publications' 是BibTeX文件的文件名

% \clearpage\end{CJK}
\end{document}


%% 文件结尾 `template-zh.tex'.
