% !TEX TS-program = XeLaTeX
% !TEX encoding = UTF-8 Unicode
%% start of file `template-zh.tex'.
%% Copyright 2006-2012 Xavier Danaux (xdanaux@gmail.com).
% 
% This work may be distributed and/or modified under the
% conditions of the LaTeX Project Public License version 1.3c,
% available at http://www.latex-project.org/lppl/.

\documentclass[10pt,a4paper,sans]{moderncv}   % possible options include font size ('10pt', '11pt' and '12pt'), paper size ('a4paper', 'letterpaper', 'a5paper', 'legalpaper', 'executivepaper' and 'landscape') and font family ('sans' and 'roman')
\usepackage{etoolbox}
\usepackage{coffee}

\makeatletter
% 字符编码
\usepackage[utf8]{inputenc}                   % 替换你正在使用的编码
% 使用xelatex,无需以下设置
% \usepackage{CJKutf8}

\@ifpackageloaded{tex4ht}{
  \usepackage{devng4ht}
}{%
  \usepackage{fontspec,xunicode}
%  \setmainfont{WenQuanYi Zen Hei Mono}
  \setmainfont{PT Mono}
  \usepackage[slantfont,boldfont]{xeCJK}
%  \setCJKmainfont{WenQuanYi Zen Hei}
  \setCJKmainfont{Heiti SC}
  \defaultfontfeatures{Mapping=tex-text}
  \XeTeXlinebreaklocale "zh"
  \XeTeXlinebreakskip = 0pt plus 1pt minus 0.1pt
}
\usepackage{xcolor}                 % replace by the encoding you are using

% moderncv 主题
\moderncvstyle{classic}                        % 选项参数是 ‘casual’, ‘classic’, ‘oldstyle’ 和 ’banking’
\moderncvcolor{green}                          % 选项参数是 ‘blue’ (默认)、‘orange’、‘green’、‘red’、‘purple’ 和 ‘grey’
% \nopagenumbers{}                             % 消除注释以取消自动页码生成功能

% 调整页面
\usepackage[scale=0.85]{geometry}
\setlength{\hintscolumnwidth}{4cm}           % 如果你希望改变日期栏的宽度
% 解决XeLaTex编译时的编码警告
\usepackage[unicode,pdfencoding=auto]{hyperref}

% 个人信息
\firstname{徐}
\familyname{至强}
\title{高级软件架构师}                      % 可选项、如不需要可删除本行
% \address{松江区龙马路185弄25号1401室}{201616,上海}             % 可选项、如不需要可删除本行
\extrainfo{出生:1983.11 年龄:38}
\mobile{+86 13585618661}                         %
\mobile{+1 (805) 6698661}                         % 可选项、如不需要可删除本行
\phone{+86 21 67663700}                          % 可选项、如不需要可删除本行
%% \fax{+3~(456)~789~012}                            % 可选项、如不需要可删除本行
\email{xeonxu@gmail.com}                    % 可选项、如不需要可删除本行
\homepage{blog.xeonxu.info}                  % 可选项、如不需要可删除本行
\social[github]{xeonxu}                  % 可选项、如不需要可删除本行
\social[twitter]{xeonxu}                  % 可选项、如不需要可删除本行
% \photo[80pt][0.8pt]{picture.jpg}                  % ‘64pt’是图片必须压缩至的高度、‘0.4pt‘是图片边框的宽度 (如不需要可调节至0pt)、’picture‘ 是图片文件的名字;可选项、如不需要可删除本行
\quote{Work as a hacker. Hack as an artist.}                 % optional, remove the line if not want

% 显示索引号;仅用于在简历中使用了引言
% \makeatletter
% \renewcommand*{\bibliographyitemlabel}{\@biblabel{\arabic{enumiv}}}
% \makeatother

% 分类索引
% \usepackage{multibib}
% \newcites{book,misc}{{Books},{Others}}
% ----------------------------------------------------------------------------------
% 内容
% ----------------------------------------------------------------------------------
\begin{document}
% \begin{CJK}{UTF8}{gbsn}                       % 详情参阅CJK文件包
\maketitle

% \cofeAm{0.2}{0.7}{180}{0cm}{0cm}
% \cofeSplash{}
\section{简介}
\cvitem{}{
  \begin{itemize}
  \item 本科毕业,学士学位。软件开发15年,国家软件设计师职称。
  \item 拥有 14 年嵌入式软件开发经验:3 年车载项目,7.5 年手机研发, 3.5 年机器人研发;其中,有5 年时间专注手机电池管理,3 年 ROS 平台应用开发,1.5 年 RT-Thread 及 STM32传感器设计开发经验。
  \item 熟悉 Linux 开发环境,内核驱动开发调试,RT-Thread 开发及配置, Android开发配置, GCC 工具链和GDB调试工具。能够根据需求,编写实现相应脚本进行自动化系统集成并构建高效的开发测试环境。
  \item 熟练使用 C 语言,Bash 脚本,Git 版本管理,Jenkins工具,以及能够基于Docker做一些开发和应用。此外,会使用python,elisp,common lisp 等。
  \item 拥有 2 年管理经验,负责过 3 款手机驱动的研发工作及机器人软件架构的重构工作。
  \item 积极参与开源项目,包括 tmk keyboard,RT-Thread,koreader,proxmark3等。
  \item 渴求且敬畏技术,作为第一发明人有7项已授权专利。
  \end{itemize}
}

\section{教育背景}
\cventry{2002.09 -- 2006.06}{交通工程学士学位}{河海大学}{南京}{}{}

\section{IT技能}
\cvdoubleitem{C语言}{熟练}{Java}{普通}
\cvdoubleitem{内核驱动}{熟练}{RT-Thread}{普通}
\cvdoubleitem{Linux/Unix}{熟练}{Android BSP}{熟练}
\cvdoubleitem{Git/Repo}{熟练}{Docker}{普通}
\cvdoubleitem{Bash}{熟练}{Python}{普通}
\cvdoubleitem{Emacs}{熟练}{Vim}{普通}
\cvdoubleitem{Latex}{普通}{Scheme}{普通}
\cvdoubleitem{Router Operating System}{普通}{Robot Operating System}{普通}

\section{语言技能}
\cvdoubleitem{普通话}{}{}{良好}
\cvdoubleitem{英语}{}{}{工作交流}
\cvdoubleitem{日语}{}{}{工作交流}

\section{工作履历}

\cventry{2019.10 -- 至今}{驱动系统架构师}{上海上汽集团商用车技术中心}{上海市}{}{
  工作内容:汽车网联系统OTA协议修订及维护,智能座舱软件架构的设计定义,重构工作。
  \newline{}
  主要工作:
  \begin{itemize}
  \item 自建网联OTA协议修订及维护工作。
  \item 智能座舱团队基础开发工具链的搭建,以及持续集成服务的配置工作。
  \item 车身控制器时钟校准策略的制定。
  \item 高通8155智能座舱项目
    \begin{itemize}
    \item QNX+Android 方案中的CAN 信号到系统接口的架构方案。
    \item 斑马L+L方案中负责空调模块开发和调试工作。
    \item SOA 信号矩阵的沟通和定义。
    \item “人机共驾”功能中,确认高精地图的编译,匹配,渲染等方面的落地方案和细节。
    \item 负责前后排地图互动的架构方案和技术细节。
    \end{itemize}  
  \end{itemize}  
}
\cventry{2018.06 -- 2019.09}{高级软件工程师 架构师}{上海云绅智能科技有限公司}{上海市}{}{
  工作内容:红星美凯龙“大天使”物联网服务机器人软件架构设计及实现,公司研发环境的构建和维护
  \newline{}
  主要工作:
  \begin{itemize}
  \item 实现STM32F4在RT Thread下的CAN驱动,编写并实现CAN分析仪在ROS下的通信节点
  \item 基于RT Thread开发传感器开发框架,定义通信协议,并开发实现上位机的ROS节点
  \item 基于CAN总线技术的传感器固件升级方案设计与实现
  \item 机器人ROS系统程序的在线升级(FOTA)功能的设计与实现
  \item 机器人系统及软件节点的标准输出和打包,实现软件审核上库后,20分钟内部署入目标机
  \item 移植配置Intel realsense的驱动和ROS节点,移植人脸识别算法节点,解决摄像头热拔插问题
  \item 制定机器人内部路由器部署方案,实现4G和wifi双上行自动链路切换功能,同时支持VPN拨号至公司内网做业务维护,另外还支持和不同网段的的机内三方设备互联互通
  \item 为全息项目开发基于UWB的室内定位方案
  \item 标准化工控机生产和部署,开发相关生产工具,实现无人为干预下5分钟生产部署一台工控系统的能力
  \item 编译实现ROS的Docker环境镜像,同时搭建公司docker镜像库服务器
  \item 搭建维护gerrit版本服务器,jenkins编译服务器,基于ROS配置AP漫游以及QOS
  \end{itemize}
}
\cventry{2016.07 -- 2018.05}{软件解决方案专家}{上海木爷机器人有限公司}{上海市}{}{
  工作内容:机器人软件研发,软件架构设计\newline{}
  主要工作:
  \begin{itemize}
  \item 开发基于RK3288芯片的Android平板,负责RK818的电池管理以及优化改进系统启动流程
  \item 基于NVidia TX1工控板开发传感器驱动,接口和主控逻辑,研发生产镜像代码管控方案
  \item 标准化X86工控板系统,提高装机效率
  \item 探索实现基于Docker下ROS的机器人运行时
  \item 标准化的ROS程序编译系统及打包流程
  \item 设计实现灵活的机器人程序自动升级架构
  \item 调研使用RTOS重构单片机开发架构,降低开发和维护成本,推动使用开源GCC构建目标程序。
  \end{itemize}
}
\cventry{2010.06 -- 2016.06}{嵌入式软件工程师}{上海斐讯数据通信技术有限公司}{上海市}{}{
  工作内容:手机研发\newline{}
  主要工作:
  \begin{itemize}
  \item 开发分别基于Marvell平台和Qualcomm平台的Android智能设备项目
  \item 开发并推广windows和linux下可一键安装配置的高通编译环境
  \item 开发推广基于ditcc及ccache的android分布缓存编译优化方案
  \item 开发传感器驱动,实现HAL和相应framework。包括地磁,加速度,距离,光以及TP等传感器设备
  \item 负责电源管理部分逻辑开发,包括电池建模,充放电策略,多路充电切换等
  \item 特别地在工作中优化改进了电池建模,BMS算法
  \item 优化基于控制台的开发环境和开发工具,极大提高项目效率
  \end{itemize}
}
\cventry{2009.10 -- 2010.06}{嵌入式软件工程师}{上海品酷网络科技有限公司}{上海市}{}{
  工作内容:开发动态手机主题及手机特效(iShow主题系统)\newline{}
  主要工作:
  \begin{itemize}
  \item 开发基于mtk平台手机的动态主题和菜单特效
  \item 可无极变色的飘带效果动态屏保
  \item 可区分昼夜的蒲公英效果动态屏保
  \item 编写研发自动化脚本工具
  \item 改进完善手机端的http下载程序,支持断点续传
  \end{itemize}
}
\cventry{2006.07 -- 2009.10}{软件工程师}{上海腾龙集团}{上海市}{}{
  \cventry{2008.11 -- 2009.10}{嵌入式软件工程师}{上海腾龙软件公司}{上海市}{}{}{}
    工作内容:对日项目,主要是嵌入式平台的开发。\newline{}
    主要工作:
    \begin{itemize}
    \item 在日立公司的数字电视项目中负责界面开发和编码测试
    \item vxworks到Linux的移植项目。改进开发环境,快速解决问题。
    \item openvg移植项目,负责主体程序移植以及单元测试编写和回归测试
    \end{itemize}
  \cventry{2008.06 -- 2008.11}{Java工程师}{京瓷Mita公司}{大阪市}{外派}{}
    工作内容:开发用于医院的纸质文档的电子化解决方案KMCapture Solution。\newline{}
    主要工作:
    \begin{itemize}
    \item 主要负责Controller,Facade以及部分的Storage模块设计,编码和测试
    \end{itemize}
  \cventry{2007.05 -- 2008.06}{嵌入式软件工程师}{日立制造所}{日立市}{外派}{}
    工作内容:开发基于ARM处理器的车载项目AVCCore。该项目后来为凯迪拉克CTS08,09车型的车载音响。\newline{}
    主要工作:
    \begin{itemize}
    \item iPod集成应用。负责车载电脑中iPod Controller和iPod CoreApp开发和维护
    \item 负责车载电脑的音频文件头解析处理的开发及维护
    \end{itemize}
  \cventry{2006.07 -- 2007.05}{Java工程师}{南大腾龙}{南京市}{}{}
    工作内容:为三得利公司开发自动售卖机在线维护管理系统。\newline{}
    主要工作:
    \begin{itemize}
    \item 编写数据库的处理代码和测试代码
    \item 编写自动化部署脚本
    \item 文档编写和维护
    \end{itemize}
}
\section{个人兴趣}
\cvdoubleitem{体育}{F1,跑步,滑板运动}{制作}{机械键盘,电子小器件}
\cvdoubleitem{电影}{纪录片,科幻片}{看书}{技术书籍,小说}

\section{专利}
  \cventry{201510477097.X}{一种登录认证方法及系统}{发明}{唯一发明人}{已授权}{}
  \cventry{201510612722.7}{一种扩展移动终端运算能力的系统}{发明}{唯一发明人}{已授权}{}
  \cventry{201410613796.8}{一种移动终端及其虚拟光驱的实现方法}{发明}{唯一发明人}{已授权}{}
  \cventry{201210368281.7}{一种耳机接口装置及基于所述耳机接口装置的控制方法}{发明}{唯一发明人}{已授权}{}
  \cventry{201510745100.1}{一种密钥加密方法及系统、电子设备}{发明}{唯一发明人}{已授权}{}
  \cventry{201210585845.2}{硬件固件独立升级系统及方法}{发明}{唯一发明人}{已授权}{}
  \cventry{201420615063.3}{一种手机进水保护装置}{实用新型}{联合发明人}{已授权}{}
  \cventry{CN201811126148.4A}{一种控制在指定区域活动的方法及机器人}{发明}{唯一发明人}{公示}{}
  \cventry{CN201811126930.6A}{一种磁场发生装置的匹配方法及系统、机器人}{发明}{唯一发明人}{公示}{}
  \cventry{CN201811126928.9A}{一种坐标校准方法及系统、机器人}{发明}{唯一发明人}{公示}{}

\section{开源项目}
\cventry{参与项目}{github}{}{}{}{
  \begin{itemize}
  \item twip: 添加proxy支持
  \item ChinaDNS-C: Bug修复及tomato编译支持
  \item Koreader: Bug修复及编译速度优化
  \item tmk\_keyboard: 独立为usb2usb增加蓝牙功能,蓝牙使用rn42模块
  \item RT Thread: Stm32f1板级支持bug的修复,编写Stm32f4 hal版的CAN驱动
  \item ChameleonMini-rebooted: 增加NTAG-213、215、216 的支持;增加对泽塔奥特曼变身器的支持和密钥监听。
  \end{itemize}
}
\cventry{个人项目}{github}{}{}{}{
  \begin{itemize}
  \item battery\_analyzer: 电池soc表自动测算工具
  \item vim\_configs: 维护的用于公司工作的vim配置
  \item csr\_tool: Dump csr芯片的脚本工具
  \item Ultramanmedal: 使用Proxmark3 制作泽塔奥特曼勋章和身份卡的Lua程序。
  \item bin2elf: 为bin文件增加elf文件头,从而可以将通过jtag dump出来的固件,方便使用支持elf的烧录程序将其重新写入设备。该工具主要用来备份国内违反GPL 不开源的Proxmark3 私改固件。
  \end{itemize}
}

\renewcommand{\listitemsymbol}{¤}             % 改变列表符号

% \section{其他 2}
% \cvlistdoubleitem{项目 1}{项目 4}
% \cvlistdoubleitem{项目 2}{项目 5\cite{book1}}
% \cvlistdoubleitem{项目 3}{}

% 来自BibTeX文件但不使用multibib包的出版物
% \renewcommand*{\bibliographyitemlabel}{\@biblabel{\arabic{enumiv}}}% BibTeX的数字标签
\nocite{*}
\bibliographystyle{plain}
\bibliography{publications}                    % 'publications' 是BibTeX文件的文件名

% 来自BibTeX文件并使用multibib包的出版物
% \section{出版物}
% \nocitebook{book1,book2}
% \bibliographystylebook{plain}
% \bibliographybook{publications}               % 'publications' 是BibTeX文件的文件名
% \nocitemisc{misc1,misc2,misc3}
% \bibliographystylemisc{plain}
% \bibliographymisc{publications}               % 'publications' 是BibTeX文件的文件名

% \clearpage\end{CJK}
\end{document}


%% 文件结尾 `template-zh.tex'.
