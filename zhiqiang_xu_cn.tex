% !TEX TS-program = XeLaTeX
% !TEX encoding = UTF-8 Unicode
%% start of file `template-zh.tex'.
%% Copyright 2006-2012 Xavier Danaux (xdanaux@gmail.com).
% 
% This work may be distributed and/or modified under the
% conditions of the LaTeX Project Public License version 1.3c,
% available at http://www.latex-project.org/lppl/.

\documentclass[10pt,a4paper,mono]{moderncv}   % possible options include font size ('10pt', '11pt' and '12pt'), paper size ('a4paper', 'letterpaper', 'a5paper', 'legalpaper', 'executivepaper' and 'landscape') and font family ('sans' and 'roman')
%\usepackage{etoolbox}
\usepackage{coffee}
% 当前日期获取
\usepackage[useregional]{datetime2}
\newcommand{\ctoday}{\number\year 年\number\month 月\number\day 日}

\usepackage{ifthen}
\newcounter{myage}
\setcounter{myage}{\the\year}
\addtocounter{myage}{-1983}
\ifthenelse{\the\month<11}{\addtocounter{myage}{-1}}{}
\ifthenelse{\the\month=11}{
  \ifthenelse{\the\day < 14}{\addtocounter{myage}{-1}}{}
}{}

% moderncv 主题
\moderncvstyle{classic}                        % 选项参数是 'casual' (default), 'classic', 'banking', 'oldstyle' and 'fancy'
\moderncvcolor{green}                          % 选项参数是 'black', 'blue' (default), 'burgundy', 'green', 'grey', 'orange', 'purple' and 'red'
%\renewcommand{\familydefault}{\sfdefault}         % to set the default font; use '\sfdefault' for the default sans serif font, '\rmdefault' for the default roman one, or any tex font name
%\nopagenumbers{}                                  % uncomment to suppress automatic page numbering for CVs longer than one page

% 调整页面
\usepackage[scale=0.85]{geometry}
\recomputelengths
% \setlength{\footskip}{136.00005pt}                 % depending on the amount of information in the footer, you need to change this value. comment this line out and set it to the size given in the warning
\setlength{\hintscolumnwidth}{3cm}                % 如果你希望改变日期栏的宽度
%\setlength{\makecvheadnamewidth}{10cm}            % for the 'classic' style, if you want to force the width allocated to your name and avoid line breaks. be careful though, the length is normally calculated to avoid any overlap with your personal info; use this at your own typographical risks...

% 解决XeLaTex编译时的编码警告
%\usepackage[unicode,pdfencoding=auto]{hyperref}

% font loading
% for luatex and xetex, do not use inputenc and fontenc
% see https://tex.stackexchange.com/a/496643
\ifxetexorluatex
  \usepackage{fontspec}
  \usepackage{unicode-math}
  \defaultfontfeatures{Ligatures=TeX}
  \setmainfont{Sarasa Term SC}
  \setsansfont{Sarasa Mono SC}
  \setmonofont{Sarasa Mono SC}
  \setmathfont{Latin Modern Math}
  %\usepackage[slantfont,boldfont]{xeCJK}
  %\defaultfontfeatures{Mapping=tex-text}
  \usepackage{xcolor}
  \XeTeXlinebreaklocale "zh"
  \XeTeXlinebreakskip = 0pt plus 1pt minus 0.1pt
\else
  \usepackage[utf8]{inputenc}
  \usepackage{CJKutf8}
  \usepackage[T1]{fontenc}
  \usepackage{lmodern}
\fi

% document language
\usepackage[english]{babel}  % FIXME: using spanish breaks moderncv

% 个人信息
\name{徐}{至强}
\title{高级软件架构师}                      % 可选项、如不需要可删除本行
\born{1983.11 -- \themyage{}岁}
% \address{松江区龙马路185弄25号1401室}{201616,上海}             % 可选项、如不需要可删除本行
\extrainfo{简历更新日期:\ctoday}
\phone[mobile]{+86~13585618661}                         %
\phone[mobile]{+1~(805)~6698661}                         % 可选项、如不需要可删除本行
\phone[fixed]{+86~21~67663700}                          % 可选项、如不需要可删除本行
%% \fax{+3~(456)~789~012}                            % 可选项、如不需要可删除本行
\email{xeonxu@gmail.com}                    % 可选项、如不需要可删除本行
\homepage{blog.xeonxu.info}                  % 可选项、如不需要可删除本行
\social[telegram]{xeonxu}                  % 可选项、如不需要可删除本行
\social[github]{xeonxu}                  % 可选项、如不需要可删除本行
\social[twitter]{xeonxu}                  % 可选项、如不需要可删除本行
\photo[80pt][0.8pt]{zhiqiang_xu.jpg}                  % ‘64pt’是图片必须压缩至的高度、‘0.4pt‘是图片边框的宽度 (如不需要可调节至0pt)、’picture‘ 是图片文件的名字;可选项、如不需要可删除本行
\quote{Work as a hacker. Hack as an artist.}                 % optional, remove the line if not want

% 显示索引号;仅用于在简历中使用了引言
%\makeatletter\renewcommand*{\bibliographyitemlabel}{\@biblabel{\arabic{enumiv}}}\makeatother
\renewcommand*{\bibliographyitemlabel}{[\arabic{enumiv}]}
%   to redefine the bibliography heading string ("Publications")
%\renewcommand{\refname}{Articles}

% 分类索引
% \usepackage{multibib}
% \newcites{book,misc}{{Books},{Others}}
% ----------------------------------------------------------------------------------
% 内容
% ----------------------------------------------------------------------------------
\begin{document}
\ifxetexorluatex
\else
\begin{CJK*}{UTF8}{gbsn}                          % to typeset your resume in Chinese using CJK
\fi
%-----       resume       ---------------------------------------------------------

\cofeCm{0.2}{0.7}{180}{0cm}{0cm}
% \cofeSplash{}

\makecvtitle

\section{简介}
\cvlistitem{本科毕业,学士学位。软件开发16年,国家软件设计师职称。}
\cvlistitem{拥有 16 年嵌入式软件开发经验:4 年车载项目,7.5 年手机研发, 3.5 年机器人研发;其中,有5 年时间专注手机电池管理,3 年 ROS 平台应用开发,1.5 年 RT-Thread 及 STM32传感器设计开发经验。}
\cvlistitem{熟悉 Linux 开发环境,内核驱动开发调试,RT-Thread 开发及配置, Android开发配置, GCC 工具链和GDB调试工具。能够根据需求,编写实现相应脚本进行自动化系统集成并构建高效的开发测试环境。}
\cvlistitem{熟练使用 C 语言,Bash 脚本,Git 版本管理,Jenkins工具,以及能够基于Docker做一些开发和应用。此外,会使用python,elisp,common lisp 等。}
\cvlistitem{拥有 2 年管理经验,负责过 3 款手机驱动的研发工作及机器人软件架构的重构工作。}
\cvlistitem{积极参与开源项目,包括 tmk keyboard,RT-Thread,koreader,proxmark3等。}
\cvlistitem{渴求且敬畏技术,作为第一发明人有8项已授权专利。}
\section{教育背景}
\cventry{2002.09 -- 2006.06}{交通工程学士学位}{河海大学}{南京}{}{}

\section{IT技能}
\setcvskilllegendcolumns[][0.61]
\cvskillplainlegend*[0.2em][了解][可操作][熟练操作][高效操作][精通]{等级指标}
\cvdoubleitem{C语言}{\cvskill{5}}{Java}{\cvskill{3}}
\cvdoubleitem{内核驱动}{\cvskill{4}}{RT-Thread}{\cvskill{4}}
\cvdoubleitem{Linux/Unix}{\cvskill{5}}{Android BSP}{\cvskill{4}}
\cvdoubleitem{Git/Repo}{\cvskill{5}}{Docker}{\cvskill{4}}
\cvdoubleitem{Bash}{\cvskill{5}}{Python}{\cvskill{3}}
\cvdoubleitem{Emacs}{\cvskill{5}}{Vim}{\cvskill{3}}
\cvdoubleitem{\LaTeX}{\cvskill{2}}{Common Lisp}{\cvskill{2}}
\cvdoubleitem{Router Operating System}{\cvskill{3}}{Robot Operating System}{\cvskill{3}}

\section{语言技能}
\cvdoubleitem{英语}{\cvskill{3}}{日语}{\cvskill{3}}

\section{工作履历}

\cventry{2022.04 -- 至今}{系统软件主任工程师}{极氪汽车(宁波杭州湾新区)有限公司}{上海市}{}{
  工作内容:汽车座舱系统SOA通信平台架构方案设计开发。
  \newline{}
  主要工作:
  \begin{itemize}
  \item 极氪智能座舱SOA通信平台中间件架构方案设计。
  \item 基于开源vsomeip和commonapi组件,集成并实现极氪SOA平台中的通信组件部分。
  \item 基于PCD移植实现EM运行管理。
  \item 为工程添加对clangd的支持,自动生成compile commands文件。
  \item 实现基于Docker的开发环境,支持单机仿真运行和在线单步调试功能。
  \end{itemize}
}
\cventry{2019.10 -- 2022.03}{驱动系统架构师}{上海上汽集团商用车技术中心}{上海市}{}{
  工作内容:汽车网联系统OTA协议修订及维护,智能座舱软件架构的设计定义,重构工作。
  \newline{}
  主要工作:
  \begin{itemize}
  \item 自建网联OTA协议修订及维护工作。
  \item 智能座舱团队基础开发工具链的搭建,以及持续集成服务的配置工作。
  \item 车身控制器时钟校准策略的制定。
  \item 基于容器的单核多系统调研及架构方案设计。
  \item 高通8155智能座舱项目
    \begin{itemize}
    \item QNX+Android 方案中的CAN 信号到系统接口的架构方案。
    \item 斑马L+L方案中负责空调模块开发和调试工作。
    \item SOA 信号矩阵的沟通和定义。
    \item “人机共驾”功能中,确认高精地图的编译,匹配,渲染等方面的落地方案和细节。
    \item 负责前后排地图互动的架构方案和技术细节。
    \end{itemize}  
  \end{itemize}  
}
\cventry{2018.06 -- 2019.09}{高级软件工程师 架构师}{上海云绅智能科技有限公司}{上海市}{}{
  工作内容:红星美凯龙“大天使”物联网服务机器人软件架构设计及实现,公司研发环境的构建和维护
  \newline{}
  主要工作:
  \begin{itemize}
  \item 实现STM32F4在RT Thread下的CAN驱动,编写并实现CAN分析仪在ROS下的通信节点
  \item 基于RT Thread开发传感器开发框架,定义通信协议,并开发实现上位机的ROS节点
  \item 基于CAN总线技术的传感器固件升级方案设计与实现
  \item 机器人ROS系统程序的在线升级(FOTA)功能的设计与实现
  \item 机器人系统及软件节点的标准输出和打包,实现软件审核上库后,20分钟内部署入目标机
  \item 移植配置Intel realsense的驱动和ROS节点,移植人脸识别算法节点,解决摄像头热拔插问题
  \item 制定机器人内部路由器部署方案,实现4G和wifi双上行自动链路切换功能,同时支持VPN拨号至公司内网做业务维护,另外还支持和不同网段的的机内三方设备互联互通
  \item 为全息项目开发基于UWB的室内定位方案
  \item 标准化工控机生产和部署,开发相关生产工具,实现无人为干预下5分钟生产部署一台工控系统的能力
  \item 编译实现ROS的Docker环境镜像,同时搭建公司docker镜像库服务器
  \item 搭建维护gerrit版本服务器,jenkins编译服务器,基于ROS配置AP漫游以及QOS
  \end{itemize}
}
\cventry{2016.07 -- 2018.05}{软件解决方案专家}{上海木爷机器人有限公司}{上海市}{}{
  工作内容:机器人软件研发,软件架构设计\newline{}
  主要工作:
  \begin{itemize}
  \item 开发基于RK3288芯片的Android平板,负责RK818的电池管理以及优化改进系统启动流程
  \item 基于NVidia TX1工控板开发传感器驱动,接口和主控逻辑,研发生产镜像代码管控方案
  \item 标准化X86工控板系统,提高装机效率
  \item 探索实现基于Docker下ROS的机器人运行时
  \item 标准化的ROS程序编译系统及打包流程
  \item 设计实现灵活的机器人程序自动升级架构
  \item 调研使用RTOS重构单片机开发架构,降低开发和维护成本,推动使用开源GCC构建目标程序。
  \end{itemize}
}
\cventry{2010.06 -- 2016.06}{嵌入式软件工程师}{上海斐讯数据通信技术有限公司}{上海市}{}{
  工作内容:手机研发\newline{}
  主要工作:
  \begin{itemize}
  \item 开发分别基于Marvell平台和Qualcomm平台的Android智能设备项目
  \item 开发并推广windows和linux下可一键安装配置的高通编译环境
  \item 开发推广基于ditcc及ccache的android分布缓存编译优化方案
  \item 开发传感器驱动,实现HAL和相应framework。包括地磁,加速度,距离,光以及TP等传感器设备
  \item 负责电源管理部分逻辑开发,包括电池建模,充放电策略,多路充电切换等
  \item 特别地在工作中优化改进了电池建模,BMS算法
  \item 优化基于控制台的开发环境和开发工具,极大提高项目效率
  \end{itemize}
}
\cventry{2009.10 -- 2010.06}{嵌入式软件工程师}{上海品酷网络科技有限公司}{上海市}{}{
  工作内容:开发动态手机主题及手机特效(iShow主题系统)\newline{}
  主要工作:
  \begin{itemize}
  \item 开发基于mtk平台手机的动态主题和菜单特效
  \item 可无极变色的飘带效果动态屏保
  \item 可区分昼夜的蒲公英效果动态屏保
  \item 编写研发自动化脚本工具
  \item 改进完善手机端的http下载程序,支持断点续传
  \end{itemize}
}
\cventry{2006.07 -- 2009.10}{软件工程师}{上海腾龙集团}{上海市}{}{
  \cventry{2008.11 -- 2009.10}{嵌入式软件工程师}{上海腾龙软件公司}{上海市}{}{}{}
    工作内容:对日项目,主要是嵌入式平台的开发。\newline{}
    主要工作:
    \begin{itemize}
    \item 在日立公司的数字电视项目中负责界面开发和编码测试
    \item vxworks到Linux的移植项目。改进开发环境,快速解决问题。
    \item openvg移植项目,负责主体程序移植以及单元测试编写和回归测试
    \end{itemize}
  \cventry{2008.06 -- 2008.11}{Java工程师}{京瓷Mita公司}{大阪市}{外派}{}
    工作内容:开发用于医院的纸质文档的电子化解决方案KMCapture Solution。\newline{}
    主要工作:
    \begin{itemize}
    \item 主要负责Controller,Facade以及部分的Storage模块设计,编码和测试
    \end{itemize}
  \cventry{2007.05 -- 2008.06}{嵌入式软件工程师}{日立制造所}{日立市}{外派}{}
    工作内容:开发基于ARM处理器的车载项目AVCCore。该项目后来为凯迪拉克CTS 08,09 车型的车载音响。\newline{}
    主要工作:
    \begin{itemize}
    \item iPod集成应用。负责车载电脑中iPod Controller和iPod CoreApp开发和维护
    \item 负责车载电脑的音频文件头解析处理的开发及维护
    \end{itemize}
  \cventry{2006.07 -- 2007.05}{Java工程师}{南大腾龙}{南京市}{}{}
    工作内容:为三得利公司开发自动售卖机在线维护管理系统。\newline{}
    主要工作:
    \begin{itemize}
    \item 编写数据库的处理代码和测试代码
    \item 编写自动化部署脚本
    \item 文档编写和维护
    \end{itemize}
}
\section{个人兴趣}
\cvdoubleitem{体育}{F1,跑步,滑板运动}{制作}{机械键盘,电子小器件}
\cvdoubleitem{电影}{纪录片,科幻片}{看书}{技术书籍,小说}

\section{专利}
  \cventry{201510477097.X}{一种登录认证方法及系统}{发明}{唯一发明人}{已授权}{}
  \cventry{201510612722.7}{一种扩展移动终端运算能力的系统}{发明}{唯一发明人}{已授权}{}
  \cventry{201410613796.8}{一种移动终端及其虚拟光驱的实现方法}{发明}{唯一发明人}{已授权}{}
  \cventry{201210368281.7}{一种耳机接口装置及基于所述耳机接口装置的控制方法}{发明}{唯一发明人}{已授权}{}
  \cventry{201510745100.1}{一种密钥加密方法及系统、电子设备}{发明}{唯一发明人}{已授权}{}
  \cventry{201210585845.2}{硬件固件独立升级系统及方法}{发明}{唯一发明人}{已授权}{}
  \cventry{201420615063.3}{一种手机进水保护装置}{实用新型}{联合发明人}{已授权}{}
  \cventry{CN201811304800.7A}{一种活动区域热力图的生成方法和服务器}{发明}{唯一发明人}{已授权}{}
  \cventry{CN201811126148.4A}{一种控制在指定区域活动的方法及机器人}{发明}{唯一发明人}{公示}{}
  \cventry{CN201811126930.6A}{一种磁场发生装置的匹配方法及系统、机器人}{发明}{唯一发明人}{公示}{}
  \cventry{CN201811126928.9A}{一种坐标校准方法及系统、机器人}{发明}{唯一发明人}{公示}{}

\section{开源项目}
\cventry{参与项目}{github}{}{}{}{
  \begin{itemize}
  \item twip: 添加proxy支持
  \item ChinaDNS-C: Bug修复及tomato编译支持
  \item Koreader: Bug修复及编译速度优化
  \item tmk\_keyboard: 独立为usb2usb增加蓝牙功能,蓝牙使用rn42模块
  \item RT Thread: Stm32f1板级支持bug的修复,编写Stm32f4 hal版的CAN驱动
  \item ChameleonMini-rebooted: 增加NTAG-213、215、216 的支持;增加对泽塔奥特曼变身器的支持和密钥监听。
  \end{itemize}
}
\cventry{个人项目}{github}{}{}{}{
  \begin{itemize}
  \item battery\_analyzer: 电池soc表自动测算工具
  \item vim\_configs: 维护的用于公司工作的vim配置
  \item csr\_tool: Dump csr芯片的脚本工具
  \item Ultramanmedal: 使用Proxmark3 制作泽塔奥特曼勋章和身份卡的Lua程序。
  \item bin2elf: 为bin文件增加elf文件头,从而可以将通过jtag dump出来的固件,方便使用支持elf的烧录程序将其重新写入设备。该工具主要用来备份国内违反GPL 不开源的Proxmark3 私改固件。
  \item wintoolset: Windows上运行工具脚本的环境。
  \item trc2asc:将PCAN抓包的TRC文件转为通用的ASC文件。
  \item asc2blf:将ASC文件转换为BLF文件。
  \item exportdrawio:将Drawio文件中所有sheet页一键导出为PNG或者SVG,PDF等文件的工具。
  \item crackmfkey:一键获取ChameleonMini的监听数据,并计算得到对应扇区的密钥
  \end{itemize}
}

% \section{其他 2}
% \cvlistdoubleitem{项目 1}{项目 4}
% \cvlistdoubleitem{项目 2}{项目 5\cite{book1}}
% \cvlistdoubleitem{项目 3}{}

% 来自BibTeX文件但不使用multibib包的出版物
% \renewcommand*{\bibliographyitemlabel}{\@biblabel{\arabic{enumiv}}}% BibTeX的数字标签
\nocite{*}
\bibliographystyle{plain}
\bibliography{publications}                    % 'publications' 是BibTeX文件的文件名

% 来自BibTeX文件并使用multibib包的出版物
% \section{出版物}
% \nocitebook{book1,book2}
% \bibliographystylebook{plain}
% \bibliographybook{publications}               % 'publications' 是BibTeX文件的文件名
% \nocitemisc{misc1,misc2,misc3}
% \bibliographystylemisc{plain}
% \bibliographymisc{publications}               % 'publications' 是BibTeX文件的文件名

\clearpage

%-----       letter       ---------------------------------------------------------
% recipient data
% \recipient{Company Recruitment team}{Company, Inc.\\123 somestreet\\some city}
% \date{January 01, 1984}
% \opening{Dear Sir or Madam,}
% \closing{Yours faithfully,}
% \enclosure[Attached]{curriculum vit\ae{}}          % use an optional argument to use a string other than "Enclosure", or redefine \enclname
% \makelettertitle

% Lorem ipsum dolor sit amet, consectetur adipiscing elit. Duis ullamcorper neque sit amet lectus facilisis sed luctus nisl iaculis. Vivamus at neque arcu, sed tempor quam. Curabitur pharetra tincidunt tincidunt. Morbi volutpat feugiat mauris, quis tempor neque vehicula volutpat. Duis tristique justo vel massa fermentum accumsan. Mauris ante elit, feugiat vestibulum tempor eget, eleifend ac ipsum. Donec scelerisque lobortis ipsum eu vestibulum. Pellentesque vel massa at felis accumsan rhoncus.

% Suspendisse commodo, massa eu congue tincidunt, elit mauris pellentesque orci, cursus tempor odio nisl euismod augue. Aliquam adipiscing nibh ut odio sodales et pulvinar tortor laoreet. Mauris a accumsan ligula. Class aptent taciti sociosqu ad litora torquent per conubia nostra, per inceptos himenaeos. Suspendisse vulputate sem vehicula ipsum varius nec tempus dui dapibus. Phasellus et est urna, ut auctor erat. Sed tincidunt odio id odio aliquam mattis. Donec sapien nulla, feugiat eget adipiscing sit amet, lacinia ut dolor. Phasellus tincidunt, leo a fringilla consectetur, felis diam aliquam urna, vitae aliquet lectus orci nec velit. Vivamus dapibus varius blandit.

% Duis sit amet magna ante, at sodales diam. Aenean consectetur porta risus et sagittis. Ut interdum, enim varius pellentesque tincidunt, magna libero sodales tortor, ut fermentum nunc metus a ante. Vivamus odio leo, tincidunt eu luctus ut, sollicitudin sit amet metus. Nunc sed orci lectus. Ut sodales magna sed velit volutpat sit amet pulvinar diam venenatis.

% Albert Einstein discovered that $e=mc^2$ in 1905.

% \[ e=\lim_{n \to \infty} \left(1+\frac{1}{n}\right)^n \]

% \makeletterclosing

\ifxetexorluatex
\else
\clearpage\end{CJK*}                              % if you are typesetting your resume in Chinese using CJK; the \clearpage is required for fancyhdr to work correctly with CJK, though it kills the page numbering by making \lastpage undefined
\fi
\end{document}


%% 文件结尾 `template-zh.tex'.
