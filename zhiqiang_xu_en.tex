% !TEX TS-program = XeLaTeX
% !TEX encoding = UTF-8 Unicode
%% start of file `template-en.tex'.
%% Copyright 2006-2012 Xavier Danaux (xdanaux@gmail.com).
%
% This work may be distributed and/or modified under the
% conditions of the LaTeX Project Public License version 1.3c,
% available at http://www.latex-project.org/lppl/.

\documentclass[10pt,a4paper,mono,final]{moderncv}   % possible options include font size ('10pt', '11pt' and '12pt'), paper size ('a4paper', 'letterpaper', 'a5paper', 'legalpaper', 'executivepaper' and 'landscape') and font family ('sans' and 'roman')
%\usepackage{etoolbox}
\usepackage{coffee}
% 当前日期获取
\usepackage[useregional]{datetime2}
\newcommand{\ctoday}{\number\month -\number\day -\number\year}

\usepackage{ifthen}
\newcounter{myage}
\setcounter{myage}{\the\year}
\addtocounter{myage}{-1983}
\ifthenelse{\the\month<11}{\addtocounter{myage}{-1}}{}
\ifthenelse{\the\month=11}{
  \ifthenelse{\the\day < 14}{\addtocounter{myage}{-1}}{}
}{}

% moderncv 主题
\moderncvstyle{classic}                        % 选项参数是 'casual' (default), 'classic', 'banking', 'oldstyle' and 'fancy'
\moderncvcolor{green}                          % 选项参数是 'black', 'blue' (default), 'burgundy', 'green', 'grey', 'orange', 'purple' and 'red'
%\renewcommand{\familydefault}{\sfdefault}         % to set the default font; use '\sfdefault' for the default sans serif font, '\rmdefault' for the default roman one, or any tex font name
%\nopagenumbers{}                                  % uncomment to suppress automatic page numbering for CVs longer than one page

% 调整页面
\usepackage[scale=0.85]{geometry}
\recomputelengths
\setlength{\footskip}{136.00005pt}                 % depending on the amount of information in the footer, you need to change this value. comment this line out and set it to the size given in the warning
\setlength{\hintscolumnwidth}{3cm}                % 如果你希望改变日期栏的宽度
%\setlength{\makecvheadnamewidth}{10cm}            % for the 'classic' style, if you want to force the width allocated to your name and avoid line breaks. be careful though, the length is normally calculated to avoid any overlap with your personal info; use this at your own typographical risks...

% 解决XeLaTex编译时的编码警告
%\usepackage[unicode,pdfencoding=auto]{hyperref}

% font loading
% for luatex and xetex, do not use inputenc and fontenc
% see https://tex.stackexchange.com/a/496643
\ifxetexorluatex
  \usepackage{fontspec}
  \usepackage{unicode-math}
  \defaultfontfeatures{Ligatures=TeX}
  \setmainfont{Sarasa Term SC}
  \setsansfont{Sarasa Mono SC}
  \setmonofont{Sarasa Mono SC}
  \setmathfont{Latin Modern Math}
  %\usepackage[slantfont,boldfont]{xeCJK}
  %\defaultfontfeatures{Mapping=tex-text}
  \usepackage{xcolor}
  \XeTeXlinebreaklocale "zh"
  \XeTeXlinebreakskip = 0pt plus 1pt minus 0.1pt
\else
  \usepackage[utf8]{inputenc}
  \usepackage{CJKutf8}
  \usepackage[T1]{fontenc}
  \usepackage{lmodern}
\fi

% document language
\usepackage[english]{babel}  % FIXME: using spanish breaks moderncv

% 个人信息
\firstname{Zhiqiang}
\familyname{Xu}
\title{Software Architect}                      % 可选项、如不需要可删除本行
%\address{Room 1401, Blvd. 25, Longma Rd. 185 Nong, Songjiang Dicstrict}{201616,Shanghai}             % 可选项、如不需要可删除本行
\born{1983.11 -- \themyage{}}
\extrainfo{Last update:\ctoday}
\phone[mobile]{+86~13585618661}                         %
\phone[mobile]{+1~(805)~6698661}                         % 可选项、如不需要可删除本行
\phone[fixed]{+86~21~67663700}                          % 可选项、如不需要可删除本行
%% \fax{+3~(456)~789~012}                            % 可选项、如不需要可删除本行
\email{xeonxu@gmail.com}                    % 可选项、如不需要可删除本行
\homepage{blog.xeonxu.info}                  % 可选项、如不需要可删除本行
\social[telegram]{xeonxu}                  % 可选项、如不需要可删除本行
\social[github]{xeonxu}                  % 可选项、如不需要可删除本行
\social[twitter]{xeonxu}                  % 可选项、如不需要可删除本行
\photo[80pt][0.8pt]{zhiqiang_xu.jpg}                  % ‘64pt’是图片必须压缩至的高度、‘0.4pt‘是图片边框的宽度 (如不需要可调节至0pt)、’picture‘ 是图片文件的名字;可选项、如不需要可删除本行
\quote{Work as a hacker. Hack as an artist.}                 % optional, remove the line if not want

% 显示索引号;仅用于在简历中使用了引言
%\makeatletter\renewcommand*{\bibliographyitemlabel}{\@biblabel{\arabic{enumiv}}}\makeatother
\renewcommand*{\bibliographyitemlabel}{[\arabic{enumiv}]}
%   to redefine the bibliography heading string ("Publications")
%\renewcommand{\refname}{Articles}

% 分类索引
%\usepackage{multibib}
%\newcites{book,misc}{{Books},{Others}}
%----------------------------------------------------------------------------------
%            内容
%----------------------------------------------------------------------------------
\begin{document}
\ifxetexorluatex
\else
\begin{CJK*}{UTF8}{gbsn}                          % to typeset your resume in Chinese using CJK
\fi
%-----       resume       ---------------------------------------------------------

\cofeCm{0.2}{0.7}{180}{0cm}{0cm}
% \cofeSplash{}

\makecvtitle


\section{PROFILE}

\cvlistitem{Bachelor's degree. 18 years of embedded software development, 2 years of management experience.}
\cvlistitem{7 years in automotive projects, 7 years in mobile phone R\&D, 4 years in robotics R\&D.}
\cvlistitem{Proficient in Linux development environment, kernel driver development and debugging.}
\cvlistitem{Skilled in using scripts to automate system integration and development environment.}
\cvlistitem{Proficient in C language, Bash script, Git version control, and capable of doing some development and application based on Docker.}
\cvlistitem{As a hobby, I enjoy Lisp and its dialects.}
\cvlistitem{Actively participate in open source projects, including tmk keyboard, RT-Thread, koreader, proxmark3, etc.}
\cvlistitem{Eager to learn and respect technology, with 8 authorized patents as the first inventor.}

\section{EDUCATIONAL QUALIFICATIONS}
\cventry{2002.09 -- 2006.06}{Bachelor of Traffic Engineering}{Hohai University}{Nanjing}{}{}  % 第3到第6编码可留白

% \section{毕业论文}
% \cvitem{题目}{\emph{地铁机车运行信息收发报模拟器的实现}}
% \cvitem{导师}{徐鹏}

\section{LANGUAGE SKILLS}
\cvitemwithcomment{English}{CET4}{Normal}
\cvitemwithcomment{Japanese}{}{Normal}

\section{TECHNICAL SKILLS AND KNOWLEDGE}
\setcvskilllegendcolumns[][0.61]
\cvskillplainlegend*[0.2em]

\cvdoubleitem{C}{\cvskill{5}}{Java}{\cvskill{3}}
\cvdoubleitem{Kernel Driver}{\cvskill{4}}{RT-Thread}{\cvskill{4}}
\cvdoubleitem{Linux/Unix}{\cvskill{5}}{Android BSP}{\cvskill{4}}
\cvdoubleitem{Git/Repo}{\cvskill{5}}{Docker}{\cvskill{4}}
\cvdoubleitem{Bash}{\cvskill{5}}{Python}{\cvskill{3}}
\cvdoubleitem{Emacs}{\cvskill{5}}{Vim}{\cvskill{3}}
\cvdoubleitem{\LaTeX}{\cvskill{2}}{Common Lisp}{\cvskill{2}}
\cvdoubleitem{Router Operating System}{\cvskill{3}}{Robot Operating System}{\cvskill{3}}

\section{PROFESSIONAL EXPERIENCE}


\cventry{2022.04 -- Present}{Senior Software Engineer}{Zeekr Automobile (Ningbo Hangzhou Bay New Area)}{Shanghai}{}{
  Responsibilities: Design and development of the SOA communication platform architecture for automotive cockpit systems.
  \begin{itemize}
  \item Designed the middleware architecture solution for the Zeekr Automotive intelligent cockpit SOA communication platform.
    \begin{itemize}
    \item Developed SOA communication components for Zeekr Automotive 2.x platform vehicles based on open-source vsomeip and commonapi.
    \item Designed and implemented a Docker-based development and compilation environment to ensure consistency in compilation environments for each target version. Provided single-machine communication simulation for SOMEIP services and online step debugging capabilities to enhance the development efficiency of SOA components.
    \item Led the team to design and implement tools for converting communication matrix design files into C++ and Java code. This allowed for quick initiation of service business development based on template projects converted by the tool during the development of native and Android programs.
    \item Led the team to design and implement the SDK of SOA components, allowing the SOA SDK to be compiled and adapted to support Linux, QNX, Android, and other operating systems, thereby achieving platformization of communication components.
    \item Developed a SOMEIP Ethernet packet parsing tool, which parsed SOMEIP Ethernet packets into service interfaces and corresponding data content based on communication matrix definition files. This tool improved the efficiency of service data confirmation to some extent during middleware development and service integration.
    \item Designed various business integration and automation tools based on the inconvenience of using development prototype interfaces. Combined with ADB and ROS routers, these tools enabled the development computer to access the prototype network, facilitating simulation integration and debugging of services.
    \end{itemize}
  \item Developed business integration tools for the ZKOS communication solution
    \begin{itemize}
    \item Verified its tool capabilities and communication framework capabilities.
    \item Wrote service interface testing programs based on the SOA service definitions of the 3.0 platform models.
    \item Developed service validation tools to simplify the process of using ZKOS for development, compilation, deployment, and running test programs.
    \item Participated in and guided the detachment solution of the ZKOS SDK.
    \end{itemize}
  \item Developed the Ethernet diagnostic architecture for the 3.0 platform and implemented its prototype. Developed Ethernet diagnostic services for the 3.0 platform.
  \end{itemize}
}
\cventry{2019.10 -- 2022.03}{Software Architect}{SAIC MAXUS}{Shanghai}{}{
  Responsibilities: Revision and maintenance of OTA protocols for automotive networking systems, design and definition of intelligent cockpit software architecture, and refactoring work.
  \begin{itemize}
  \item Self-built networking OTA protocol revision and maintenance work.
  \item Establishment of the basic development toolchain for the intelligent cockpit team and configuration of continuous integration services.
  \item Formulation of clock calibration strategies for domain controllers.
  \item Research and architectural design of container-based single-core multi-system solutions.
  \item Qualcomm 8155 intelligent cockpit project
    \begin{itemize}
    \item Architectural design of CAN signal to system interface in the QNX+Android solution.
    \item Responsible for the development and debugging of the air conditioning module in the Zebra L+L solution.
    \item Communication and definition of SOA signal matrices.
    \item Confirmation of design details of HD map compilation, matching, rendering, etc.
    \item Architectural design and technical details communication for co-operating navimap between front IVI and rear IVI.
    \end{itemize} 
  \end{itemize}  
}
\cventry{2018.06 -- 2019.09}{Senior Software Engineer}{Shanghai Yunshen technology Co., Ltd. }{Shanghai}{}{Responsibilities: Robot software architecture design and implementation, and construction and maintenance of company R\&D environment.
  \begin{itemize}
  \item Sensor development
    \begin{itemize}
    \item Implemented CAN driver for STM32F4 under RT Thread, and wrote and implemented communication nodes for CAN analyzer under ROS.
    \item Developed a sensor development framework based on RT Thread, defined communication protocols, and developed ROS nodes for the host computer.
    \item Designed and implemented a sensor firmware upgrade solution based on CAN bus technology.
    \item Developed a UWB-based indoor positioning solution for holographic projects.
    \end{itemize}
  \item Robot control system
    \begin{itemize}
    \item Standardized output and packaging of robot system and software nodes, enabling deployment into target machines within 20 minutes after software update and approval.
    \item Ported and configured Intel Realsense drivers and ROS nodes, ported facial recognition algorithm nodes, and resolved camera hot plug issues.
    \item Construct system image for central controller, which is also used for factory production.
    \item Formulated deployment plans for internal routers of robots, implemented automatic link switching between 4G and Wi-Fi uplinks, and supported VPN dialing to the company's intranet for business maintenance. Additionally, supported interconnection between internal third-party devices with different sub-network settings.
    \item Designed and implemented the online upgrade (FOTA) function of robot ROS system programs.
    \end{itemize}
  \item System management
    \begin{itemize}
    \item Compiled and implemented ROS Docker environment images, and built a company Docker image library server.
    \item Established and maintained Gerrit version servers, Jenkins compilation servers.
    \item Configured AP roaming and QoS based on ROS (Router Operating System).
    \end{itemize}
  \end{itemize}
}
\cventry{2016.07 -- 2018.05}{Software Solution Expert}{Shanghai clever mrobot technologies Co., Ltd. }{Shanghai}{}{Responsibilities: Robot software development and software architecture design.
  \begin{itemize}
  \item Developed Android tablets based on RK3288 chips, responsible for RK818 battery management and optimizing system startup processes.
  \item Developed sensor drivers, interfaces, and main control logic based on the NVidia TX1 industrial control board.
  \item Standardized system install method for control board to improve installation efficiency.
  \item Explored the implementation of robot runtime based on ROS under Docker.
  \item Standardized ROS program compilation systems and packaging processes.
  \item Designed and implemented a flexible architecture for automatic robot program upgrades.
  \item Researched and used RTOS to refactor MCU development architectures, reducing development and maintenance costs, and promoting the use of open-source GCC to build target programs.
  \end{itemize}
}
\cventry{2010.06 -- 2016.06}{Embedded Software Engineer}{Shanghai Phicomm. Inc.}{Shanghai}{}{Responsibilities: Mobile phone development
  \begin{itemize}
  \item Developed Android smart device projects based on Marvell and Qualcomm platforms.
  \item Developed and promoted one-click installation and configuration of Qualcomm compilation environments for both Windows and Linux.
  \item Implemented Docker packaging for a complete Android compilation environment to provide consistent compilation configuration and environment.
  \item Developed and promoted Android distributed caching compilation optimization solutions based on ditcc and ccache.
  \item Developed sensor drivers and implemented HAL and corresponding frameworks. This includes gyroscope, magnetic, gsensor, psensor, lsensor, and touch panel (TP) sensor devices.
  \item Responsible for the logic development of power management, including battery modeling, charging and discharging strategies, and multi-channel charging switching.
  \item Optimized and improved battery modeling and BMS algorithms in particular during work.
  \item Optimized console-based development environments and development tools, greatly improving project efficiency.
  \end{itemize}
}
\cventry{2009.10 -- 2010.06}{Embedded Software Engineer}{Shanghai iBingo network technology Co., Ltd.}{Shanghai}{}{Responsibilities: Development of dynamic mobile phone themes and special effects (iShow theme system).
  \begin{itemize}
  \item Developed dynamic themes and menu effects for mobile phones based on the MTK platform.
  \item Implemented a dynamic screensaver with unlimited color-changing streamer effects.
  \item Created a dynamic screensaver with dandelion effects that distinguish between day and night.
  \item Developed scripts for auto compiling, auto packaging, and auto releasing.
  \item Improved and perfected the HTTP download program on mobile phones, supporting resume downloads.
  \end{itemize}
}
\cventry{2008.11 -- 2009.10}{Embedded Software Engineer}{Shanghai Dragontec Group}{Shanghai}{}{Responsibilities:Japanese outsourcing job.
  \begin{itemize}
  \item Digital TV for Hitachi development.
    \begin{itemize}
    \item Responsible for interface development and code testing in Hitachi's digital TV project.
    \item Ported from VxWorks to Linux. Improved development environment and quickly resolved issues.
    \item Ported OpenVG project, responsible for main program porting, unit test code writing, and regression testing.
    \end{itemize}
  \end{itemize}
}
\cventry{2008.06 -- 2008.11}{Java Engineer}{Kyrocera Mita}{Oosaka}{}{Responsibilities: Developed KMCapture Solution, an electronic document solution for hospitals.
  \begin{itemize}
  \item Designed, coded, and tested Controller, Facade, and part of the Storage module.
  \end{itemize}
}
\cventry{2007.05 -- 2008.06}{Embedded Software Engineer}{Hitachi ICS}{Hitachi}{}{Responsibilities: Developed AVCCore, a vehicle project based on ARM processors, later used in the Cadillac CTS 08, 09 models' car audio systems.
  \begin{itemize}
  \item Integrated iPod application, responsible for developing and maintaining iPod Controller and iPod CoreApp in the vehicle computer.
  \item Developed and maintained audio file header parsing in the vehicle computer.
  \end{itemize}
}
\cventry{2006.07 -- 2007.05}{Java Software Engineer}{Nanking University Dragontec}{Nanking}{}{Responsibilities: Developed an online maintenance management system for vending machines for Suntory.
  \begin{itemize}
  \item Wrote database processing code and testing code.
  \item Wrote automation deployment scripts.
  \end{itemize}
}%

\section{HOBBIES}
% \cvitem{Movie}{\small Documentary, Sci-Fi}
% \cvitem{Reading}{\small Technology, Novell}
% \cvitem{Sports}{\small Skate Board, F1, Marathon}
\cvdoubleitem{Reading}{\small Technology, Novell}{Sports}{\small Skate Board, F1, Marathon}

\section{GRANTED PATENTS}
\cvitem{201510477097.X}{Login authentication method and system}
\cvitem{201510612722.7}{System for expanding operational capability of mobile terminal}
\cvitem{201410613796.8}{Mobile terminal and achievement method of virtual drive of same}
\cvitem{201210368281.7}{Headphone interface device and control method based on the headphone interface device}
\cvitem{201510745100.1}{A kind of key encryption method and system, electronic equipment}
\cvitem{201210585845.2}{Hardware and firmware independently updating system and method thereof}
\cvitem{201420615063.3}{A kind of mobile phone water enter preventing device}
\cvitem{CN201811304800.7A}{Generation method of thermodynamic diagram of active area and server}
\cventry{CN201811126928.9A}{Coordinate calibration method and system and robot}

\section{MISCELLANOUS}
\cventry{Contributed 3rd party projects}{github}{}{}{}{
  \begin{itemize}
  \item twip: Added proxy support.
  \item ChinaDNS-C: Fixed bugs and ported to tomato.
  \item Koreader: Fixed bugs and optimized compilation speed.
  \item tmk\_keyboard: Added bluetooth support for usb2usb.
  \item RT Thread: Fixed STM32F1 bsp bugs. Wrote CAN driver for Stm32f4 using hal library.
  \item ChameleonMini-rebooted: Added NTAG-213、215、216 support.
  \item lsp-sonarlint: An Emacs plugin from using sonarlint as a lsp backend to help checking codes rules. Added c/c++ checking support.
  \end{itemize}
}

\cventry{My projects}{github}{}{}{}{
  \begin{itemize}
  \item vim\_configs: Vim configure used for co-workers.
  \item battery\_analyzer:  A tool that auto sampling and use spline method to generate SOC table for battery driver
  \item csr\_tool: Script which can dump firmware from CSR chips.
  \item Ultramanmedal: A lua script which can generate tags for zeta ultraman.
  \item bin2elf: Add elf header for bin file.
  \item wintoolset: An environment for running script tools under windows.
  \item trc2asc:Convert trc file which is saved by peak apps to asc file.
  \item asc2blf:Convert ASC to blf or blf to asc.
  \item exportdrawio:Export png svg pdf files from a drawio file.
  \item crackmfkey:Auto download sniff data from chameleon and calcuate keys out.
  \end{itemize}
}
% \item images2bin
% \item rasproxy

\renewcommand{\listitemsymbol}{-}             % 改变列表符号

% \section{其他 2}
% \cvlistdoubleitem{项目 1}{项目 4}
% \cvlistdoubleitem{项目 2}{项目 5\cite{book1}}
% \cvlistdoubleitem{项目 3}{}

% 来自BibTeX文件但不使用multibib包的出版物
% \renewcommand*{\bibliographyitemlabel}{\@biblabel{\arabic{enumiv}}}% BibTeX的数字标签
\nocite{*}
\bibliographystyle{plain}
\bibliography{publications}                    % 'publications' 是BibTeX文件的文件名

% 来自BibTeX文件并使用multibib包的出版物
% \section{出版物}
% \nocitebook{book1,book2}
% \bibliographystylebook{plain}
% \bibliographybook{publications}               % 'publications' 是BibTeX文件的文件名
% \nocitemisc{misc1,misc2,misc3}
% \bibliographystylemisc{plain}
% \bibliographymisc{publications}               % 'publications' 是BibTeX文件的文件名

\clearpage

% -----       letter       ---------------------------------------------------------
% recipient data
% \recipient{Company Recruitment team}{Company, Inc.\\123 somestreet\\some city}
% \date{January 01, 1984}
% \opening{Dear Sir or Madam,}
% \closing{Yours faithfully,}
% \enclosure[Attached]{curriculum vit\ae{}}          % use an optional argument to use a string other than "Enclosure", or redefine \enclname
% \makelettertitle

% Lorem ipsum dolor sit amet, consectetur adipiscing elit. Duis ullamcorper neque sit amet lectus facilisis sed luctus nisl iaculis. Vivamus at neque arcu, sed tempor quam. Curabitur pharetra tincidunt tincidunt. Morbi volutpat feugiat mauris, quis tempor neque vehicula volutpat. Duis tristique justo vel massa fermentum accumsan. Mauris ante elit, feugiat vestibulum tempor eget, eleifend ac ipsum. Donec scelerisque lobortis ipsum eu vestibulum. Pellentesque vel massa at felis accumsan rhoncus.

% Suspendisse commodo, massa eu congue tincidunt, elit mauris pellentesque orci, cursus tempor odio nisl euismod augue. Aliquam adipiscing nibh ut odio sodales et pulvinar tortor laoreet. Mauris a accumsan ligula. Class aptent taciti sociosqu ad litora torquent per conubia nostra, per inceptos himenaeos. Suspendisse vulputate sem vehicula ipsum varius nec tempus dui dapibus. Phasellus et est urna, ut auctor erat. Sed tincidunt odio id odio aliquam mattis. Donec sapien nulla, feugiat eget adipiscing sit amet, lacinia ut dolor. Phasellus tincidunt, leo a fringilla consectetur, felis diam aliquam urna, vitae aliquet lectus orci nec velit. Vivamus dapibus varius blandit.

% Duis sit amet magna ante, at sodales diam. Aenean consectetur porta risus et sagittis. Ut interdum, enim varius pellentesque tincidunt, magna libero sodales tortor, ut fermentum nunc metus a ante. Vivamus odio leo, tincidunt eu luctus ut, sollicitudin sit amet metus. Nunc sed orci lectus. Ut sodales magna sed velit volutpat sit amet pulvinar diam venenatis.

% Albert Einstein discovered that $e=mc^2$ in 1905.

% \[ e=\lim_{n \to \infty} \left(1+\frac{1}{n}\right)^n \]

% \makeletterclosing

\ifxetexorluatex
\else
  \clearpage\end{CJK*}                              % if you are typesetting your resume in Chinese using CJK; the \clearpage is required for fancyhdr to work correctly with CJK, though it kills the page numbering by making \lastpage undefined
\fi
\end{document}


%% 文件结尾 `template-en.tex'.
