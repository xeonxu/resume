% !TEX TS-program = XeLaTeX
% !TEX encoding = UTF-8 Unicode
%% start of file `template-en.tex'.
%% Copyright 2006-2012 Xavier Danaux (xdanaux@gmail.com).
%
% This work may be distributed and/or modified under the
% conditions of the LaTeX Project Public License version 1.3c,
% available at http://www.latex-project.org/lppl/.

\documentclass[10pt,a4paper,mono,final]{moderncv}   % possible options include font size ('10pt', '11pt' and '12pt'), paper size ('a4paper', 'letterpaper', 'a5paper', 'legalpaper', 'executivepaper' and 'landscape') and font family ('sans' and 'roman')
%\usepackage{etoolbox}
\usepackage{coffee}
% 当前日期获取
\usepackage[useregional]{datetime2}
\newcommand{\ctoday}{\number\month -\number\day -\number\year}

\usepackage{ifthen}
\newcounter{myage}
\setcounter{myage}{\the\year}
\addtocounter{myage}{-1983}
\ifthenelse{\the\month<11}{\addtocounter{myage}{-1}}{}
\ifthenelse{\the\month=11}{
  \ifthenelse{\the\day < 14}{\addtocounter{myage}{-1}}{}
}{}

% moderncv 主题
\moderncvstyle{classic}                        % 选项参数是 'casual' (default), 'classic', 'banking', 'oldstyle' and 'fancy'
\moderncvcolor{green}                          % 选项参数是 'black', 'blue' (default), 'burgundy', 'green', 'grey', 'orange', 'purple' and 'red'
%\renewcommand{\familydefault}{\sfdefault}         % to set the default font; use '\sfdefault' for the default sans serif font, '\rmdefault' for the default roman one, or any tex font name
%\nopagenumbers{}                                  % uncomment to suppress automatic page numbering for CVs longer than one page

% 调整页面
\usepackage[scale=0.85]{geometry}
\recomputelengths
\setlength{\footskip}{136.00005pt}                 % depending on the amount of information in the footer, you need to change this value. comment this line out and set it to the size given in the warning
\setlength{\hintscolumnwidth}{3cm}                % 如果你希望改变日期栏的宽度
%\setlength{\makecvheadnamewidth}{10cm}            % for the 'classic' style, if you want to force the width allocated to your name and avoid line breaks. be careful though, the length is normally calculated to avoid any overlap with your personal info; use this at your own typographical risks...

% 解决XeLaTex编译时的编码警告
%\usepackage[unicode,pdfencoding=auto]{hyperref}

% font loading
% for luatex and xetex, do not use inputenc and fontenc
% see https://tex.stackexchange.com/a/496643
\ifxetexorluatex
  \usepackage{fontspec}
  \usepackage{unicode-math}
  \defaultfontfeatures{Ligatures=TeX}
  \setmainfont{Sarasa Term SC}
  \setsansfont{Sarasa Mono SC}
  \setmonofont{Sarasa Mono SC}
  \setmathfont{Latin Modern Math}
  %\usepackage[slantfont,boldfont]{xeCJK}
  %\defaultfontfeatures{Mapping=tex-text}
  \usepackage{xcolor}
  \XeTeXlinebreaklocale "zh"
  \XeTeXlinebreakskip = 0pt plus 1pt minus 0.1pt
\else
  \usepackage[utf8]{inputenc}
  \usepackage{CJKutf8}
  \usepackage[T1]{fontenc}
  \usepackage{lmodern}
\fi

% document language
\usepackage[english]{babel}  % FIXME: using spanish breaks moderncv

% 个人信息
\firstname{Zhiqiang}
\familyname{Xu}
\title{Senior software architect}                      % 可选项、如不需要可删除本行
%\address{Room 1401, Blvd. 25, Longma Rd. 185 Nong, Songjiang Dicstrict}{201616,Shanghai}             % 可选项、如不需要可删除本行
\born{1983.11 -- \themyage{}}
\extrainfo{Last update:\ctoday}
\phone[mobile]{+86~13585618661}                         %
\phone[mobile]{+1~(805)~6698661}                         % 可选项、如不需要可删除本行
\phone[fixed]{+86~21~67663700}                          % 可选项、如不需要可删除本行
%% \fax{+3~(456)~789~012}                            % 可选项、如不需要可删除本行
\email{xeonxu@gmail.com}                    % 可选项、如不需要可删除本行
\homepage{blog.xeonxu.info}                  % 可选项、如不需要可删除本行
\social[telegram]{xeonxu}                  % 可选项、如不需要可删除本行
\social[github]{xeonxu}                  % 可选项、如不需要可删除本行
\social[twitter]{xeonxu}                  % 可选项、如不需要可删除本行
\photo[80pt][0.8pt]{zhiqiang_xu.jpg}                  % ‘64pt’是图片必须压缩至的高度、‘0.4pt‘是图片边框的宽度 (如不需要可调节至0pt)、’picture‘ 是图片文件的名字;可选项、如不需要可删除本行
\quote{Work as a hacker. Hack as an artist.}                 % optional, remove the line if not want

% 显示索引号;仅用于在简历中使用了引言
%\makeatletter\renewcommand*{\bibliographyitemlabel}{\@biblabel{\arabic{enumiv}}}\makeatother
\renewcommand*{\bibliographyitemlabel}{[\arabic{enumiv}]}
%   to redefine the bibliography heading string ("Publications")
%\renewcommand{\refname}{Articles}

% 分类索引
%\usepackage{multibib}
%\newcites{book,misc}{{Books},{Others}}
%----------------------------------------------------------------------------------
%            内容
%----------------------------------------------------------------------------------
\begin{document}
\ifxetexorluatex
\else
\begin{CJK*}{UTF8}{gbsn}                          % to typeset your resume in Chinese using CJK
\fi
%-----       resume       ---------------------------------------------------------

\cofeCm{0.2}{0.7}{180}{0cm}{0cm}
% \cofeSplash{}

\makecvtitle


\section{SUMMARY}
\cvlistitem{B.E. with 17 years work experience and 2 years management experience.}
\cvlistitem{17 years of experience in embedded software development, with a diverse background that includes 6 years in automotive projects. 7 more years in mobile device R\&D, 4 years in robot R\&D. Notably, I spent 5 years specializing in mobile battery management, 3 years in ROS application development, and 1 years in sensor FW development using STM32 with RT-Thread. }
\cvlistitem{Well-versed in Linux development environments. Familiar with kernel driver development and debugging. Proficient in using scripts for automating system integration and development environments.}
\cvlistitem{Proficient in C language, bash scripting, version control and docker etc.. Like using lisp and its dialects as hobbies.}
\cvlistitem{Actively involving in open-source projects, including TMK Keyboard, RT-Thread, KOReader, Proxmark3, etc.}
\cvlistitem{Passionate and reverent towards technology. 8 granted patents as the first inventor.}

\section{PROFESSIONAL EXPERIENCE}
\cventry{2022.04 -- Present}{Principal Software Engineer}{Zeekr Automotive (Ningbo Hangzhou Bay New Area)}{Shanghai}{}{
  Responsibilities: Design and develop SOA communication platform for cockpit system.
  \begin{itemize}
  \item Design SOA middleware for Zeekr 2nd-Gen platform
    \begin{itemize}
    \item Design and implement SOA communication component for Zeekr 2nd-Gen platform based on open-sourced vsomeip and commonapi, enabling communication within and between domain controllers using SOMEIP.
    \item Design and implement a development and compilation environment based on Docker to ensure consistency across target version compilation environments. Provide capabilities for single-machine simulation and online debugging to enhance SOA component development efficiency.
    \item Led team to design and implement a tool suit that converts matrix files into C++ and Java template codes. Which enables the rapid development of SOA applications for both native and android programs.
    \item Led team to implement SOA SDK. Which can be compiled and adapted to supoort various OSs, including Linux, QNX, Android, and more. This initiative successfully achieved platformization of the SOA SDK.
    \item Implement a tool for generating wireshark dessector plugins from SOA matrix files, which facilitating middleware development and service integration testing by simplifying service data verification.
    \item Design and implement a solution for connectting prototypes to PCs with ethernet by using usb-ethernet adaptor and a ROS router. This solution is easy to use, scalable and cheaper than T1-Tx adapter method.
    \end{itemize}
  \item Zeekr 3rd-Gen ZKOS Communication Solution
    \begin{itemize}
    \item Verify the capabilities of the ZKOS communication solution and communication framework tools.
    \item Write service interface test programs based on the SOA service definitions of the 3.0 vehicle platform and validate the communication integrity in the latest version of ZKOS.
    \item Develop service verification tools to streamline the process of ZKOS development, compilation, deployment, and testing.
    \item Extract the ZKOS SDK from the original ZKOS toolchain.
    \end{itemize}
  \end{itemize}
}
\cventry{2019.10 -- 2022.03}{Software architect}{SAIC MAXUS}{Shanghai}{}{
  Responsibilities: Software architecture design for next generation cockpit.
  \begin{itemize}
  \item OTA protocol maintenance and revision.
  \item CI/CD services deployment.
  \item Domain controllers' clock sync policy constitution.
  \item Multi-system running on single-core H/W scheme design.
  \item Software modules definition and architecture design for intelligent cockpit controller based on Qualcomm 8155.
    \begin{itemize}
    \item Design and review the solution to extract signals from CAN to HLOS.
    \item Air condition application development for Banma OS.
    \item SOA matrix design an review.
    \item HD map compliling, matching, and rendering achitecture for AD.
    \item Design artitecture for co-operation Navi map between front and rear IVI.
    \end{itemize} 
  \end{itemize}  
}
\cventry{2018.06 -- 2019.09}{Senior System architect}{Shanghai Yunshen technologies Co., Ltd. }{Shanghai}{}{Responsibilities: robotic development
  \begin{itemize}
  \item Sensor development
    \begin{itemize}
    \item rt-thread CAN driver implementation for STM32.
    \item Ultrasonic, anti-collision, anti-falling sensors development.
    \item Sensor's firmware upgrading over CAN method implementation.
    \end{itemize}
  \item Robot control system
    \begin{itemize}
    \item Package ROS running environment by docker image.
    \item Integrate camera driver and CAN adapter driver into docker image.
    \item ROS nodes for devices and applications implementation.
    \item Build system image for x86.
    \item Mikrotik router configuration.
    \item FOTA function for X86 based controller implementation.
    \end{itemize}
  \item System management
    \begin{itemize}
    \item Docker registry server.
    \item LDAP, Git, Gerrit, VPN and Jenkins server construction and management.
    \item QoS configuration for ROS (Router Operating System).
    \end{itemize}
  \end{itemize}
}
\cventry{2016.07 -- 2018.05}{Senior Embedded Software Engineer}{Shanghai clever mrobot technologies Co., Ltd. }{Shanghai}{}{Responsibilities: Robotic development
  \begin{itemize}
  \item Android driver development
    \begin{itemize}
    \item Rockwell 3288 BSP development.
    \item System stability and charging system optimization.
    \end{itemize}
  \item Robot control system
    \begin{itemize}
    \item Robot control system redesign.
    \item Construct newly robot control system by ROS (Robot Operating System).
    \item Porting robot control system from x86 to Nvidia TX1. Integrate OpenCV support into ROS. 
    \end{itemize}
  \end{itemize}
}
\cventry{2010.06 -- 2016.06}{Embedded Software Engineer}{Shanghai Phicomm. Inc.}{Shanghai}{}{Responsibilities: Mobile phone development
  \begin{itemize}
  \item Android Smartphone project
    \begin{itemize}
    \item On Marvell PXA968 platform. Sensor driver and PMIC driver development.
    \item Qualcomm MSM7x27/MSM8x25/MSM8x25Q platform. PMIC module development. Develop environment optimization: geomagnetic debug method optimization, TP fw upgrade method. Dual-way charging method implementation based on smb358. Batch review tool for gerrit.
    \item Qualcomm MSM8x10/8x12/8x26/8x16/8994/8909 platform. PMIC module, battery characterization.
    \item Package all-in-one environment to a single self-uncompressable file for convenient quick setting for newbie developers under windows and linux.
    \item Package Android compile environment by docker technique to standardize compile environment.
    \item Optimize android compilation efficiency by using ditcc and ccache.
    \end{itemize}
  \item MStar and MTK feature phone development
    \begin{itemize}
    \item MMI customization, SP porting for customer.
    \item Scripting for auto release.
    \end{itemize}
  \end{itemize}
}
\cventry{2009.10 -- 2010.06}{Embedded software engineer}{Shanghai iBingo network technology Co., Ltd.}{Shanghai}{}{Responsibilities: Develop dynamic theme and screen saver for feature phone (iShow theme system).
  \begin{itemize}
  \item Dynamic themes development for feature phone.
    \begin{itemize}
    \item 3X3 style, Ring style theme.
    \item Ribbon style screensaver.
    \item Dandelion screensaver which can distinguish between daylight and night.
    \item Scripting for auto release.
    \end{itemize}
  \item Optimize downloading efficiency. 
  \end{itemize}
}
\cventry{2008.11 -- 2009.10}{Embedded Software Engineer}{Shanghai Dragontec Group}{Shanghai}{}{Responsibilities:Japanese outsourcing job.
  \begin{itemize}
  \item Digital TV for Hitachi development.
    \begin{itemize}
    \item UI implementation and bug fix.
    \item Documents composing.
    \item Coding and testing.
    \end{itemize}
  \item System porting from vxworks to Linux.
  \item Openvg porting project.
    \begin{itemize}
    \item Port opencv to SuperH platform for Hitachi television project.
    \item Unit test cases.
    \end{itemize}
  \end{itemize}
}
\cventry{2008.06 -- 2008.11}{Java Engineer}{Kyrocera Mita}{Oosaka}{}{Responsibilities: Non-paper solution development for hospital.
  \begin{itemize}
  \item Controller module, Facade module and Storage module implementation.
  \item Documents composing.
  \item Coding and testing.
  \end{itemize}
}
\cventry{2007.05 -- 2008.06}{Embedded Software Engineer}{Hitachi ICS}{Hitachi}{}{Responsibilities: Development of IVI for Cadillac CTS 08 and 09.
  \begin{itemize}
  \item iPod support functions development. In charge of maintenance of coreapp task and controller task.
  \item Audio file header parser development.
  \item Documents composing. 
  \end{itemize}
}
\cventry{2006.07 -- 2007.05}{Java Software Engineer}{Nanking University Dragontec}{Nanking}{}{Responsibilities: Suntory vending machine online management system development.
  \begin{itemize}
  \item Constructing sql for operating databases under DB2 and PostgreSQL.
  \item Scripting for software release and deployment.
  \item Unit test cases.
  \item Documents composing.
  \end{itemize}
}%

\section{EDUCATIONAL QUALIFICATIONS}
\cventry{2002.09 -- 2006.06}{BA of Traffic engineering}{Hohai University}{Nanking}{}{}  % 第3到第6编码可留白

% \section{毕业论文}
% \cvitem{题目}{\emph{地铁机车运行信息收发报模拟器的实现}}
% \cvitem{导师}{徐鹏}

\section{LANGUAGE SKILLS}
\cvitemwithcomment{English}{CET4}{Normal}
\cvitemwithcomment{Japanese}{}{Normal}

\section{TECHNICAL SKILLS AND KNOWLEDGE}
\setcvskilllegendcolumns[][0.61]
\cvskillplainlegend*[0.2em]

\cvdoubleitem{C}{\cvskill{5}}{Java}{\cvskill{3}}
\cvdoubleitem{Kernel Driver}{\cvskill{4}}{RT-Thread}{\cvskill{4}}
\cvdoubleitem{Linux/Unix}{\cvskill{5}}{Android BSP}{\cvskill{4}}
\cvdoubleitem{Git/Repo}{\cvskill{5}}{Docker}{\cvskill{4}}
\cvdoubleitem{Bash}{\cvskill{5}}{Python}{\cvskill{3}}
\cvdoubleitem{Emacs}{\cvskill{5}}{Vim}{\cvskill{3}}
\cvdoubleitem{\LaTeX}{\cvskill{2}}{Common Lisp}{\cvskill{2}}
\cvdoubleitem{Router Operating System}{\cvskill{3}}{Robot Operating System}{\cvskill{3}}

\section{HOBBIES}
% \cvitem{Movie}{\small Documentary, Sci-Fi}
% \cvitem{Reading}{\small Technology, Novell}
% \cvitem{Sports}{\small Skate Board, F1, Marathon}
\cvdoubleitem{Reading}{\small Technology, Novell}{Sports}{\small Skate Board, F1, Marathon}

\section{GRANTED PATENTS}
\cvitem{201510477097.X}{一种登录认证方法及系统}
\cvitem{201510612722.7}{一种扩展移动终端运算能力的系统}
\cvitem{201410613796.8}{一种移动终端及其虚拟光驱的实现方法}
\cvitem{201210368281.7}{一种耳机接口装置及基于所述耳机接口装置的控制方法}
\cvitem{201510745100.1}{一种密钥加密方法及系统、电子设备}
\cvitem{201210585845.2}{硬件固件独立升级系统及方法}
\cvitem{201420615063.3}{一种手机进水保护装置}
\cvitem{CN201811304800.7A}{一种活动区域热力图的生成方法和服务器}
\cventry{CN201811126928.9A}{一种坐标校准方法及系统、机器人}

\section{MISCELLANOUS}
\cventry{3rd party projects}{github}{}{}{}{
  \begin{itemize}
  \item twip: Add proxy support.
  \item ChinaDNS-C: Fix bug and port to tomato.
  \item Koreader: Bug fix and compilation speed optimize.
  \item tmk\_keyboard: Add bluetooth support for usb2usb.
  \item RT Thread: Fix STM32F1 bsp bugs. Write CAN driver for Stm32f4 using hal library.
  \item ChameleonMini-rebooted: Add NTAG-213、215、216 support.
  \item lsp-sonarlint: An Emacs plugin from using sonarlint as a lsp backend to help checking codes rules. Add c/c++ checking support.
  \end{itemize}
}

\cventry{Own projects}{github}{}{}{}{
  \begin{itemize}
  \item battery\_analyzer: Use spline method to generate battery SOC table.
  \item vim\_configs: Vim configure used for co-workers.
  \item ChameleonMini-rebooted: Add support for NTAG-213、215、216.
  \item battery\_analyzer: A tool that can auto sampling and calculate SOC table for battery.
  \item csr\_tool: Script which can dump firmware from CSR chips.
  \item Ultramanmedal: A lua script which can generate tags for zeta ultraman.
  \item bin2elf: Add elf header for bin file.
  \item wintoolset: An environment for running script tools under windows.
  \item trc2asc:Convert trc file which is saved by peak apps to asc file.
  \item asc2blf:Convert ASC to blf or blf to asc.
  \item exportdrawio:Export png svg pdf files from a drawio file.
  \item crackmfkey:Auto download sniff data from chameleon and calcuate keys out.
  \end{itemize}
}
% \item images2bin
% \item rasproxy

\renewcommand{\listitemsymbol}{-}             % 改变列表符号

% \section{其他 2}
% \cvlistdoubleitem{项目 1}{项目 4}
% \cvlistdoubleitem{项目 2}{项目 5\cite{book1}}
% \cvlistdoubleitem{项目 3}{}

% 来自BibTeX文件但不使用multibib包的出版物
% \renewcommand*{\bibliographyitemlabel}{\@biblabel{\arabic{enumiv}}}% BibTeX的数字标签
\nocite{*}
\bibliographystyle{plain}
\bibliography{publications}                    % 'publications' 是BibTeX文件的文件名

% 来自BibTeX文件并使用multibib包的出版物
% \section{出版物}
% \nocitebook{book1,book2}
% \bibliographystylebook{plain}
% \bibliographybook{publications}               % 'publications' 是BibTeX文件的文件名
% \nocitemisc{misc1,misc2,misc3}
% \bibliographystylemisc{plain}
% \bibliographymisc{publications}               % 'publications' 是BibTeX文件的文件名

\clearpage

% -----       letter       ---------------------------------------------------------
% recipient data
% \recipient{Company Recruitment team}{Company, Inc.\\123 somestreet\\some city}
% \date{January 01, 1984}
% \opening{Dear Sir or Madam,}
% \closing{Yours faithfully,}
% \enclosure[Attached]{curriculum vit\ae{}}          % use an optional argument to use a string other than "Enclosure", or redefine \enclname
% \makelettertitle

% Lorem ipsum dolor sit amet, consectetur adipiscing elit. Duis ullamcorper neque sit amet lectus facilisis sed luctus nisl iaculis. Vivamus at neque arcu, sed tempor quam. Curabitur pharetra tincidunt tincidunt. Morbi volutpat feugiat mauris, quis tempor neque vehicula volutpat. Duis tristique justo vel massa fermentum accumsan. Mauris ante elit, feugiat vestibulum tempor eget, eleifend ac ipsum. Donec scelerisque lobortis ipsum eu vestibulum. Pellentesque vel massa at felis accumsan rhoncus.

% Suspendisse commodo, massa eu congue tincidunt, elit mauris pellentesque orci, cursus tempor odio nisl euismod augue. Aliquam adipiscing nibh ut odio sodales et pulvinar tortor laoreet. Mauris a accumsan ligula. Class aptent taciti sociosqu ad litora torquent per conubia nostra, per inceptos himenaeos. Suspendisse vulputate sem vehicula ipsum varius nec tempus dui dapibus. Phasellus et est urna, ut auctor erat. Sed tincidunt odio id odio aliquam mattis. Donec sapien nulla, feugiat eget adipiscing sit amet, lacinia ut dolor. Phasellus tincidunt, leo a fringilla consectetur, felis diam aliquam urna, vitae aliquet lectus orci nec velit. Vivamus dapibus varius blandit.

% Duis sit amet magna ante, at sodales diam. Aenean consectetur porta risus et sagittis. Ut interdum, enim varius pellentesque tincidunt, magna libero sodales tortor, ut fermentum nunc metus a ante. Vivamus odio leo, tincidunt eu luctus ut, sollicitudin sit amet metus. Nunc sed orci lectus. Ut sodales magna sed velit volutpat sit amet pulvinar diam venenatis.

% Albert Einstein discovered that $e=mc^2$ in 1905.

% \[ e=\lim_{n \to \infty} \left(1+\frac{1}{n}\right)^n \]

% \makeletterclosing

\ifxetexorluatex
\else
  \clearpage\end{CJK*}                              % if you are typesetting your resume in Chinese using CJK; the \clearpage is required for fancyhdr to work correctly with CJK, though it kills the page numbering by making \lastpage undefined
\fi
\end{document}


%% 文件结尾 `template-en.tex'.
