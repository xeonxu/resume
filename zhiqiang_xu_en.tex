% !TEX TS-program = XeLaTeX
% !TEX encoding = UTF-8 Unicode
%% start of file `template-zh.tex'.
%% Copyright 2006-2012 Xavier Danaux (xdanaux@gmail.com).
%
% This work may be distributed and/or modified under the
% conditions of the LaTeX Project Public License version 1.3c,
% available at http://www.latex-project.org/lppl/.


\documentclass[10pt,a4paper,sans]{moderncv}   % possible options include font size ('10pt', '11pt' and '12pt'), paper size ('a4paper', 'letterpaper', 'a5paper', 'legalpaper', 'executivepaper' and 'landscape') and font family ('sans' and 'roman')
\usepackage{etoolbox}
\usepackage{coffee}

\makeatletter
% 字符编码
\usepackage[utf8]{inputenc}                   % 替换你正在使用的编码
% 使用xelatex,无需以下设置
% \usepackage{CJKutf8}

\@ifpackageloaded{tex4ht}{
\usepackage{devng4ht}
}{%
\usepackage{fontspec,xunicode}
% \setmainfont{DejaVu Serif}
\setmainfont{PragmataPro for Powerline}
\usepackage[slantfont,boldfont]{xeCJK}
% \setCJKmainfont{WenQuanYi Zen Hei}
\setCJKmainfont{STXihei}
\defaultfontfeatures{Mapping=tex-text}
\XeTeXlinebreaklocale "zh"
\XeTeXlinebreakskip = 0pt plus 1pt minus 0.1pt
}
\usepackage{xcolor}                 % replace by the encoding you are using

% moderncv 主题
\moderncvstyle{classic}                        % 选项参数是 ‘casual’, ‘classic’, ‘oldstyle’ 和 ’banking’
\moderncvcolor{green}                          % 选项参数是 ‘blue’ (默认)、‘orange’、‘green’、‘red’、‘purple’ 和 ‘grey’
%\nopagenumbers{}                             % 消除注释以取消自动页码生成功能

% 调整页面
\usepackage[scale=0.85]{geometry}
\setlength{\hintscolumnwidth}{4cm}           % 如果你希望改变日期栏的宽度
% 解决XeLaTex编译时的编码警告
\usepackage[unicode,pdfencoding=auto]{hyperref}

% 个人信息
\firstname{Zhiqiang}
\familyname{Xu}
\title{Senior software architect}                      % 可选项、如不需要可删除本行
%\address{Room 1401, Blvd. 25, Longma Rd. 185 Nong, Songjiang Dicstrict}{201616,Shanghai}             % 可选项、如不需要可删除本行
\extrainfo{Birth:1983.11 Age:35}
\mobile{+86 135 8561 8661}                         % 可选项、如不需要可删除本行
\mobile{+1 805 669 8661}                         % 可选项、如不需要可删除本行
\phone{+86 21 6766 3700}                          % 可选项、如不需要可删除本行
%% \fax{+3~(456)~789~012}                            % 可选项、如不需要可删除本行
\email{xeonxu@gmail.com}                    % 可选项、如不需要可删除本行
\homepage{blog.xeonxu.info}                  % 可选项、如不需要可删除本行
\social[github]{xeonxu}                  % 可选项、如不需要可删除本行
\social[twitter]{xeonxu}                  % 可选项、如不需要可删除本行
% \photo[80pt][0.8pt]{picture.jpg}                  % ‘64pt’是图片必须压缩至的高度、‘0.4pt‘是图片边框的宽度 (如不需要可调节至0pt)、’picture‘ 是图片文件的名字;可选项、如不需要可删除本行
% \quote{Work as a hacker. Hack as an artist.}                 % optional, remove the line if not want

% 显示索引号;仅用于在简历中使用了引言
%\makeatletter
%\renewcommand*{\bibliographyitemlabel}{\@biblabel{\arabic{enumiv}}}
%\makeatother

% 分类索引
%\usepackage{multibib}
%\newcites{book,misc}{{Books},{Others}}
%----------------------------------------------------------------------------------
%            内容
%----------------------------------------------------------------------------------
\begin{document}
% \begin{CJK}{UTF8}{gbsn}                       % 详情参阅CJK文件包
\maketitle

\cofeAm{0.2}{0.7}{180}{0cm}{0cm}
\section{SUMMARY}
\cvitem{}{
  BA. With 13 years embeded software develop expriences. Including 2yrs on vehicle related projects, 5yrs on mobile phone projects, and 3yrs on robot related projects.\newline
  Excellent in C, bash, linux, rt-thread, version control, docker.\newline
  Be familiar with software design, problem solving and debuging skills.\newline
  Experience on refactoring robot control system using ROS and docker technic. Using RTOS to implement STM32 based sensors.\newline
  2yrs experience on team management. \newline
  % \begin{itemize}
  % \item BA. With 13 years embeded software develop expriences. Including 2yrs on vehicle related projects, 5yrs on mobile phone projects, and 3yrs on robot related projects.
  % \item Excellent in C, bash, linux, rt-thread, version control, docker.
  % \item Be familiar with software design, problem solving and debuging skills.
  % \item Experience on refactoring robot control system using ROS and docker technic. Using RTOS to implement STM32 based sensors.
  % \item 2yrs experience on team management. 
  % \end{itemize}
}

\section{PROFESSIONAL EXPERIENCE}
\cventry{2019.10 -- present}{System architect}{SAIC MAXUS}{Shanghai}{}{
  Work content: Design software architect for next generation AVN.
  \newline{}
  Main work:
  \begin{itemize}
  \item SV5101 AVN architecture.
  \end{itemize}  
}
\cventry{2018.06 -- 2019.09}{Senior System architect}{Shanghai Yunshen technologies Co., Ltd. }{Shanghai}{}{Work content: robot development\newline{}
Main work:
\begin{itemize}
 \item Sensor development
  \begin{itemize}
  \item implement rt-thread CAN driver for STM32.
  \item Develop ultrasonic, anti-colllision, anti-falling sensors. Report sensors' infomations through CAN bus.
  \item Design firmware update method for sensors through CAN bus.
  \end{itemize}
 \item Robot control system
  \begin{itemize}
  \item Use docker image to package ROS enviroment.
  \item Integrate camera driver and CAN adaptor driver in docker image.
  \item Write ROS node to connect between drivers and applications.
  \item Export clone system method for producing in factories. 
  \item Router configuration for robot.
  \item Use repo method to manage robot softwares.
  \item Implement OTA function for X86 based controller.
  \end{itemize}
 \item System management
  \begin{itemize}
  \item Docker registry server.
  \item LDAP, Git, gerrit, vpn and jenkins server management.
  \item ROS(Router Operating System)'s QoS configuration.
  \end{itemize}
\end{itemize}
  }
\cventry{2016.07 -- 2018.05}{Senior Embeded Software Engineer}{Shanghai clever mrobot technologies Co., Ltd. }{Shanghai}{}{Work content: robot development\newline{}
Main work:
\begin{itemize}
 \item Android driver development
  \begin{itemize}
  \item Based on Rockwell 3288 platform. Solving charging issues.
  \item Stability optimization.
  \end{itemize}
 \item Robot control system
  \begin{itemize}
  \item Redesign robot control system using ROS.
  \item Use ROS(Robot Operating System) to construct control system for robotic products.
  \item Integrate OpenCV support on TX1.
  \end{itemize}
\end{itemize}
}
\cventry{2010.06 -- 2016.06}{Embeded Software Engineer}{Shanghai Phicomm. Inc.}{Shanghai}{}{Work content: Mobile phone development\newline{}
Main work:
\begin{itemize}
 \item Android Smartphone project
  \begin{itemize}
  \item Develop on Marvell PXA968 platform. Sensors' and PMIC's drivers. Logo file generator.
  \item Qualcomm MSM7x27/MSM8x25/MSM8x25Q platform. PMIC module, develop enviroment optimize. Mainly, PMIC module development, geomagnetic debug method optimization, TP fw upgrade method and dual-way charging method based on smb358. Batch review tool for gerrit.
  \item Qualcomm MSM8x10/8x12/8x26/8x16/8994/8909 platform. PMIC module, battery charactorization.
  \end{itemize}
\item MStar and MTK feature phone project
   \begin{itemize}
  \item MMI custimization, add SP for customer.
  \item Write auto release script.
  \end{itemize}
\end{itemize}
}
\cventry{2009.10 -- 2010.06}{Embeded software engineer}{Shanghai iBingo network technology Co., Ltd.}{Shanghai}{}{Work content: Develop dynamic theme and screen saver for feature phone (iShow theme system).\newline{}
  Main work:
  \begin{itemize}
  \item Develop dynamic themes for feature phone base on MTK platform.
    \begin{itemize}
    \item 3X3 style, Ring style theme.
    \item Ribbon style screensaver.
    \item Dandelion screensaver which can distinction between day and light.
    \item Auto release script.
    \end{itemize}
  \item Optimize downloading efficiency. 
  \end{itemize}
}
\cventry{2008.11 -- 2009.10}{Embeded Software Engineer}{Shanghai Dragontec Group}{Shanghai}{}{Work content:Japanese outsourcing software project.\newline{}
  Main work:
  \begin{itemize}
  \item Digital TV develop project from Hitachi.
    \begin{itemize}
    \item UI implement and bug fixment.
    \item Write documentation.
    \item Code and test.
    \end{itemize}
  \item System port from vxworks to Linux. Develop enviroment improvement.
  \item Openvg port project.
    \begin{itemize}
    \item Port opencv support to SuperH platform for Hitachi television project.
    \item Write unit test cases.
    \end{itemize}
  \end{itemize}
}
\cventry{2008.06 -- 2008.11}{Java Engineer}{Kyrocera Mita}{Oosaka}{}{Work content: Use Java to develop a non-paper work solution for hospital, which is called KMCapture Solution. \newline{}
  Main work:
  \begin{itemize}
  \item Work for implementing Controller, Facade and Storage modules.
  \item Write documentation.
  \item Code and test.
  \end{itemize}
}
\cventry{2007.05 -- 2008.06}{Embeded Software Engineer}{Hitachi ICS}{Hitachi}{}{Work content: Work for developing a car audio equipment for Cadillac CTS 08 and 09. \newline{}
  Main work:
  \begin{itemize}
  \item iPod interface and control system.
  \item Parser for getting informations from audio file.
  \item Write documentation. 
  \end{itemize}
}
\cventry{2006.07 -- 2007.05}{Java Software Engineer}{Nanking University Dragontec}{Nanking}{}{Work content:Develop online management system for suntory's vending machines.\newline{}
  Main work:
  \begin{itemize}
  \item Write database process using sql.
  \item Auto release and deployment script.
  \item Write unit test cases.
  \item Write documentation.
  \end{itemize}
}%

\section{EDUCATIONAL QUALIFICATIONS}
\cventry{2002.09 -- 2006.06}{BA of Traffic engineering}{Hohai University}{Nanking}{}{}  % 第3到第6编码可留白

% \section{毕业论文}
% \cvitem{题目}{\emph{地铁机车运行信息收发报模拟器的实现}}
% \cvitem{导师}{徐鹏}

\section{LANGUAGE SKILLS}
\cvitemwithcomment{English}{CET4}{Normal}
\cvitemwithcomment{Japanese}{}{Normal}

\section{TECHNICAL SKILLS AND KNOWLEDGES}
\cvdoubleitem{C}{Proficient}{Java}{Poor}
% \cvdoubleitem{Objective-C}{Learning}{Swift}{Learning}
\cvdoubleitem{Linux/Unix}{Proficient}{Hardware}{Ease soldering}
\cvdoubleitem{Script}{Bash}{Editor}{Emacs/Vim}

\section{HOBBIES}
% \cvitem{Movie}{\small Documentary, Sci-Fi}
% \cvitem{Reading}{\small Technology, Novell}
% \cvitem{Sports}{\small Skate Board, F1, Marathon}
\cvdoubleitem{Reading}{\small Technology, Novell}{Sports}{\small Skate Board, F1, Marathon}

\section{GRANTED PANTENTS}
\cvitem{201510477097.X}{\small 一种登录认证方法及系统}
\cvitem{201510612722.7}{\small 一种扩展移动终端运算能力的系统}
\cvitem{201410613796.8}{\small 一种移动终端及其虚拟光驱的实现方法}
\cvitem{201210368281.7}{\small 一种耳机接口装置及基于所述耳机接口装置的控制方法}
\cvitem{201510745100.1}{\small 一种密钥加密方法及系统、电子设备}
\cvitem{201210585845.2}{\small 硬件固件独立升级系统及方法}
\cvitem{201420615063.3}{\small 一种手机进水保护装置}

\section{MISCELLANOUS}
\cvlistitem{twip: Add proxy support.}
\cvlistitem{ChinaDNS-C: Bug fix and add compile target for tomato.}
\cvlistitem{Koreader: Bug fix and compilation speed optimize. }
\cvlistitem{tmk\_keyboard:Add bluetooth support for usb2usb.}
\cvlistitem{RT Thread:Fix STM32F1 bsp bugs. Write CAN driver for Stm32f4 using hal library.}
\cvlistitem{battery\_analyzer: Use spline to generate battery SOC table.}
\cvlistitem{vim\_configs: Vim configure used for co-workers.}
% \cvlistitem{images2bin}
% \cvlistitem{rasproxy}

\renewcommand{\listitemsymbol}{-}             % 改变列表符号

% \section{其他 2}
% \cvlistdoubleitem{项目 1}{项目 4}
% \cvlistdoubleitem{项目 2}{项目 5\cite{book1}}
% \cvlistdoubleitem{项目 3}{}

% 来自BibTeX文件但不使用multibib包的出版物
%\renewcommand*{\bibliographyitemlabel}{\@biblabel{\arabic{enumiv}}}% BibTeX的数字标签
\nocite{*}
\bibliographystyle{plain}
\bibliography{publications}                    % 'publications' 是BibTeX文件的文件名

% 来自BibTeX文件并使用multibib包的出版物
% \section{出版物}
% \nocitebook{book1,book2}
% \bibliographystylebook{plain}
%\bibliographybook{publications}               % 'publications' 是BibTeX文件的文件名
%\nocitemisc{misc1,misc2,misc3}
%\bibliographystylemisc{plain}
%\bibliographymisc{publications}               % 'publications' 是BibTeX文件的文件名

% \clearpage\end{CJK}
\end{document}


%% 文件结尾 `template-zh.tex'.
